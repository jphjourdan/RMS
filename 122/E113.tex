\begin{enonce}
\begin{exercise}[ID={RMS122-2 E113 X MP},subtitle={},tags={}]
On se place dans $\R^3$ muni de sa structure affine euclidienne canonique.
Soient $\cD$ une droite de $\R^3$ et $A$ un point n'appartenant pas à $\cD$.
Déterminer les surfaces régulières $\cS$ de $\R^3$ telle que, en tout point $M$ de $\cS$, le plan tangent en $M$ à $\cS$ contient $A$ et la normale en $M$ à $\cS$ coupe $\cD$.
\end{exercise}
\begin{solution}
\end{solution}
\end{enonce}
