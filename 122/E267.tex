\begin{enonce}
\begin{exercise}[ID={RMS 122 E267},subtitle={},tags={}, difficulty={0}]
\begin{enumerate}
  \item Donner un exemple de fonction continue, non identiquement nulle au voisinage de $0$ et telle que $0$ n'est pas un zéro isolé.

  \item Soient $f:\R\to\R$ dérivable et $a\in\R$.
	On suppose que $f(a)=0$ et que $a$ n'est pas un zéro isolé de $f$.
	Montrer que $f'(a)=0$.

  \item Soient $(a,b)\in\R^2$ avec $a<b$, $f:[a,b]\to\R$ dérivable telle que
	\begin{align*}
	  f(a)=f(b)&=0&
	  &\et&
	  \forall x\in]a,b[, f(x)\geq 0.
	\end{align*}
	Montrer que $f'(a) f'(b)\leq 0$.\\

	Soient $I$ un intervalle de $\R$, $p$ et $q$ dans $\CC^0(I,\R)$ et \eqref{eq:37454595} l'équation différentielle
	\begin{equation}
	  \tag{E} \label{eq:37454595}
	  y''+py'+qy=0.
	\end{equation}

  \item Soit $f$ une fonction non identiquement nulle de \eqref{eq:37454595}. Montrer que les zéros de $f$ sont isolés.

  \item Soient $f$ et $g$ deux solutions de \eqref{eq:37454595} et $t_0\in I$.
	On suppose qu'il existe $c\in\R$ tel que $f(t_0)=cg(t_0)$ et $f'(t_0)=cg'(t_0)$.
	Montrer que $f=cg$.

  \item Soient $f$ et $g$ deux solutions indépendantes de \eqref{eq:37454595}.
	Montrer que le wronskien $W$ de $f$ et de $g$ ne s'annule pas.
	Exprimer $W(t)$ en fonction de $W(t_0)$.
	Montrer que, entre deux zéros consécutifs de $f$, la fonctions $g$ s'annule.
\end{enumerate}
\end{exercise}
\begin{solution}
\end{solution}
\end{enonce}
