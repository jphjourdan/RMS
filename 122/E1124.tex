\begin{enonce}
%@title = {Maple}
\begin{exercise}[ID={RMS 122-2 E1124 Centrale PC},subtitle={},tags={}]
On se place dans $\R^2$ muni de sa structure euclidienne canonique.
Si $n\geq 2$, soit $\cC_n$ la courbe en polaire d'équation
\begin{equation*}
  \rho = \frac{\sin(n\theta)}{\sin\left( (n-1)\theta \right)}.
\end{equation*}
\begin{enumerate}
  \item Étudier $\cC_2$.
  \item Donner une expression de la courbure en un point de $\cC_n$.
  \item Tracer les courbes $\cC_2$, $\cC_3$, $\cC_4$, $\cC_5$.
    Calculer pour chacune de ces courbes, l'aire de la boucle.
\end{enumerate}
\end{exercise}
\begin{solution}
\end{solution}
\end{enonce}
