\begin{enonce}
\begin{exercise}[ID={RMS 122-2 E1310 ENSEA PC},subtitle={},tags={}, difficulty={0}]
Soit $f:P\in\R[X]\mapsto P(X+1)-P(X)$.
\begin{enumerate}
  \item Montrer que $f$ est linéaire, déterminer $\ker f$ et $\im f$.
  \item Montrer que si $Q\in\R[X]$, alors il existe un unique $P$ tel que $f(P)=Q$ et $P(0)=0$.\\
    Simplifier alors $\sum\limits_{k=0}^n Q(k)$.
  \item Calculer $\sum\limits_{k=0}^n k^2$. Généraliser.
\end{enumerate}
\end{exercise}
\begin{solution}
\end{solution}
\end{enonce}
