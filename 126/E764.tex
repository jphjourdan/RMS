\begin{enonce}
\begin{exercise}[ID={RMS126 E764},subtitle={Centrale PSI 2015},tags={}]
Soient $E$ l'espace vectoriel des suites réelles bornées et $F$ l'espace vectoriel des suites réelles dont la série associée est absolument convergente.
Si $u\in E$, on pose $N_E(u)=\sup_{n\in\N}\abs{u_n}$ ;
si $v\in F$, on pose $\tilde N_F(v) = \sum_{n=0}^{+\infty}\abs{v_n}$.
\begin{enumerate}
  \item Quelle est la relation d'inclusion entre $E$ et $F$?
    Ces espaces sont-ils de dimension finie?

  \item On note pour $v\in F$, $T_v:E\to \R, \xi\mapsto\sum_{n=0}^{+\infty} \xi_n v_n$,
    et pour $u\in E$, $\tilde T_u : F\to\R, \tilde\xi\mapsto\sum_{n=0}^{+\infty} \tilde\xi_n u_n$.

    Montrer que ces applications sont bien définies, linéaires et lipschitziennes.
\end{enumerate}
\end{exercise}
\begin{solution}
\end{solution}
\end{enonce}
