\begin{enonce}
  \glissant EXO FOIREUX ?
\begin{exercise}[ID={RMS126 E869},subtitle={CCP MP 2015},tags={}]
Soient $p\in\N$ fixé et, pour $n\in\N$, $S_n=\int_0^{+\infty} \frac{t^p}{e^t + 1} e^{-nt}\d t$.
\begin{enumerate}
  \item Montrer l'existence de $S_n$ pour tout $n\in\N$.
  \item On pose, pour $a,b\in\N*$,
    \begin{equation*}
      T(a,b)=\int_0^{+\infty} t^a e^{-bt}\d t.
    \end{equation*}
    Simplifier l'expression de $T(a,b)$.

  \item Montrer que
    \begin{equation*}
      \forall n\in\N, S_0 = (p+1)! \sum_{k=1}^{n} \frac1{k^2} + S_n.
    \end{equation*}

  \item Montrer que la suite $\left( S_n \right)$ converge.

  \item Montrer que $S_0  = (p+1)! \sum_{k=1}^{+\infty}\frac1{k^2}$.
\end{enumerate}
\end{exercise}
\begin{solution}
\end{solution}
\end{enonce}
