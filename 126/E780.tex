\begin{enonce}
\begin{exercise}[ID={RMS126 E780},subtitle={Centrale PSI 2015},tags={}, difficulty={0}]
Soit $X$ un ensemble fini.
On dit que $f:X\to X$ est une involution de $X$ si $f\circ f=\id$.
On note pour $n\in\N$, $I_n$ le nombre d'involutions de $\ito{n}$ et l'on convient que $I_0=1$.
\begin{enumerate}
  \item 
    Calculer $I_1$, $I_2$ et $I_3$.

  \item
    Montrer
    \begin{equation*}
      \forall n\in\N, I_{n+2} = I_{n+1} + (n+1) I_n.
    \end{equation*}
\end{enumerate}
Soit $S:z\mapsto\sum_{n=0}^{+\infty} \frac{I_n}{n!}z^n$.
\begin{enumerate}[resume]
  \item 
    Montrer que $S$ a un rayon $R > 0$.

  \item 
    Soit $x\in\left]-R, R\right[$.
    Calculer $(1+x)S(x)$ et en déduire une expression simple de $S(x)$.
    En déduire enfin une expression de $I_n$.
\end{enumerate}
\end{exercise}
\begin{solution}
\end{solution}
\end{enonce}
