\begin{enonce}
\begin{exercise}[ID={RMS123 E589 Mines PSI},subtitle={},tags={}]
On note $\ell^\infty\left( \R \right)$ l'ensemble des suites réelles bornées et $\ell^1\left( \R \right)$ l'ensemble des suites réelles dont la série est absolument convergente.
Pour $a\in\ell^\infty\left( \R \right)$ et $u\in\ell^1\left( \R \right)$, on note
\begin{equation*}
  \left( a,u \right) = \sum_{n=0}^\infty a_nu_n.
\end{equation*}
\begin{enumerate}
  \item Justifier l'existence de $\left( a,u \right)$.
  \item On fixe $u\in\ell^1\left( \R \right)$ et on pose $\phi_u : a\in\ell^\infty\left( \R \right)\mapsto \left( a,u \right)$.
    Montrer que l'on définit ainsi une application linéaire continue pour $\norm{*}_\infty$ ; calculer la norme subordonnée de $\phi_u$.
  \item On fixe $a\in\ell^\infty\left( \R \right)$ et on pose $\psi_a : u\in\ell^1\left( \R \right)\mapsto \left( a,u \right)$.
    Montrer que l'on définit ainsi une application linéaire continue pour $\norm{*}_1$ ; calculer la norme subordonnée de $\psi_a$.
\end{enumerate}
\end{exercise}
\begin{solution}
\end{solution}
\end{enonce}
