\begin{enonce}
\begin{exercise}[ID={Cahier RMS6 17 (X),question a.},subtitle={},tags={}]
Soit $B=\CC^0\left( [0,1],\C \right)$. Pour $x,y\in B$, on pose
\begin{equation*}
    x \star y : t\mapsto \int_0^t x(t-s) y(s)\d s.
\end{equation*}
La norme utilisé dans la suite sur $B$ est la norme
\begin{equation*}
    x\mapsto \norm{x}=\sup_{t\in[0,1]} \abs*{x(t)}.
\end{equation*}
Montrer que le produit $\star$ est une loi de composition interne sur $B$ commutative et associative. Admet-t-elle un élément neutre ?

Montrer que $\norm{x\star y}\leq\norm{x} \cdot\norm{y}$.
Montrer que $\lim_{n\to\infty} \norm*{x^n}^{1/n}=0$ (où $x^n=x\star x\star\dotsm\star x$).
\end{exercise}
\begin{solution}
\end{solution}
\end{enonce}
