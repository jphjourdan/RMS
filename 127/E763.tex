\begin{enonce}
\begin{exercise}[ID={RMS127 E763},subtitle={Mines-Ponts PSI 2016},tags={}]
%%% Collectionneur de vignettes
  Une urne contient $n\geq2$ boules distinctes $B_1,\dots,B_n$, que l'on tire successivement et avec remise.
  Soit $Y_r$ la variable aléatoire qui donne le rang du tirage au bout duquel $B_1,\dots,B_r$ ont été tirées au moins une fois.
  \begin{enumerate}
    \item Déterminer la loi, l'espérance, la variance de $Y_1$.

    \item Préciser $Y_r(\Omega)$.
      Que valent $P\left( Y_r = r \right)$ et $P\left( Y_r = r+1 \right)$?

    \item On fixe $r$.
      Pour tout $i\in\set{1,\dots,r}$, on note $W_i$ la variable aléatoire représentant le nombre de tirages nécessaires pour que, pour la première fois, $i$ boules distinctes parmi les boules $B_1,\dots,B_r$ soient sorties (ainsi, $W_r=Y_r)$.
      On pose $X_1=W_1$ et $X_i=W_i - W_{i-1}$ si $i\geq 2$.

      Déterminer la loi de $X_i$ ainsi que son espérance.

    \item En déduire l'espérance de $Y_n$.
      Trouver un équivalent de $E\left( Y_n \right)$ quand $n\to+\infty$.
  \end{enumerate}
\end{exercise}
\begin{solution}
\end{solution}
\end{enonce}
