\begin{enonce}
\begin{exercise}[ID={RMS127 E1144},subtitle={Centrale PC 2016},tags={}]
Une urne contient $2n$ boules.
Parmi ces boules, $n$ portent le numéro $0$, les $n$ autres portent les numéros de $1$ à $n$.
On tire $n$ boules de l'urne.
  Si $i\in\ito{n}$, soit $X_i$ la variable aléatoire égale à $1$ si la boule portant le numéro $i$ a été tirée, à $0$ sinon.
  \begin{enumerate}
    \item Déterminer la loi de $X_i$, la covariance $\cov(X_i, X_j)$ si $1\leq i<j\leq n$.

    \item Soit $S$ la somme des numéros tirés.
      Déterminer $E(S)$ et $V(S)$.
  \end{enumerate}
\end{exercise}
\begin{solution}
\end{solution}
\end{enonce}
