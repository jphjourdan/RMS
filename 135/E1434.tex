\begin{enonce}
\begin{exercise}[ID={RMS135 E1434},subtitle={IMT MP 2024},tags={},difficulty={}]
Cet énoncé me semble Faux!


Soit $(e_1, e_2, e_3)$ la base canonique de $\R^3$ euclidien.
Donner la matrice de la rotation $R$ autour de la droite $D$ d’équation 
\begin{equation*}
  D: x - y + z = x + y + z = 0
  \end{equation*}
et telle que
\begin{equation*}
R(e_1) = \frac{1}{\sqrt{2}}(e_1 + e_3).
  \end{equation*}
\end{exercise}
\begin{solution}
  Un vecteur directeur de $D$ est $e_1' = (1, 0, -1)$, qu’on normalise :
  \[ u = \frac{1}{\sqrt{2}}(1, 0, -1).  \]

  On a $R(e_1)\in u^\perp$, on devrait donc avoir $e_1 \in u^\perp$, ce qui n'est pas le cas\dots
\end{solution}
\end{enonce}
