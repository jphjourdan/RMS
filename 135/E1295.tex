\begin{enonce}
\begin{exercise}[ID={RMS135 E1295},subtitle={Centrale PSI 2024},tags={oraux},difficulty={}]
  Soit $H_n=\left( a_{i,j} \right)_{1\leq i,j\leq n}$ avec $a_{i,j} = \frac{1}{i+j-1}$.

  \begin{enumerate}
    \item Pour $X=\begin{pmatrix} x_1& \cdots & x_n \end{pmatrix}^T \in\R^n$, on note $q(X) = X^T H_n X$.

        Montrer que
        \begin{equation*}
          q(X) = \int_0^1 \left( x_1 + x_2 t + \dots + x_n t^{n-1} \right)^2 \d t.
        \end{equation*}
        En déduire que $H_n \in\sym_n^{++}(\R)$.

    \item On note $\lambda_n$ la plus petit valeur propre de $H_n$ et $\mu_n$ la plus grande.

      Montrer que $n\lambda_n \leq \sum\limits_{k=1}^n a_{k,k} \leq n \mu_n$.
      En déduire $\lim\limits_{n\to+\infty} \lambda_n = 0$.


    \item Montrer que
      \begin{equation*}
        \forall X\in\R^n, \lambda_n \norm{X}_2^2 \leq q(X) \leq \mu_n \norm{X}_2^2.
      \end{equation*}
  \end{enumerate}
\end{exercise}
\begin{solution}
\end{solution}
\end{enonce}
