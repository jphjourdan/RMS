\begin{enonce}
\begin{exercise}[ID={RMS135 E972},subtitle={Mines-Ponts PSI 2024},tags={oraux},difficulty={}]
On considère initialement une urne contenant une boule blanche et une boule rouge.
On tire une boule, on note sa couleur, on la remet dans l'urne et on rajoute deux boules de la même couleur que cette tirée.
On répète indéfiniment le processus.
\begin{enumerate}
  \item Calculer la probabilité de ne tirer que des boules rouges lors des $n$ premiers tirages?

  \item \label{q:RMS135E972b} Calculer la probabilité de tirer indéfiniment uniquement des boules rouges.

  \item Calculer la probabilité de tirer une boule blanche au $42$-ème tirage.

  \item Le résultat de la question~\ref{q:RMS135E972b} reste-t-il vrai si on rajoute $3$ boules (au lieu de $2$)? $4$ boules?.
\end{enumerate}
\end{exercise}
\begin{solution}
\end{solution}
\end{enonce}
