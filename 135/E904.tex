\begin{enonce}
\begin{exercise}[ID={RMS135 E904},subtitle={Mines-Ponts PSI 2024},tags={oraux},difficulty={}]
  On munit $\R_n[X]$ du produit scalaire $\PS{P}{Q} = \int_0^1 PQ$.
  On pose pour tout $P\in\R_n[X]$,
  \begin{equation*}
    \left( u(P) \right)(x) = \int_0^1 (x+t)^n P(t) \d t.
  \end{equation*}

  \begin{enumerate}
    \item Montrer que $u$ est un endomorphisme auto-adjoint de $\R_n[X]$.
      Qu'en déduit-on?

    \item Montrer que $u$ est un isomorphisme.
  \end{enumerate}

  Soit $(P_0,\dots,P_n)$ une base orthonormée de vecteurs propres de $u$ associés aux valeurs propres $\lambda_0,\dots,\lambda_n$.

  \begin{enumerate}[resume]
    \item Montrer que, pour tout $(x,y)\in\R^2$,
      \begin{equation*}
        (x+y)^n = \sum_{k=0}^n \lambda_k P_k(x) P_k(y).
      \end{equation*}

    \item En déduire que $\tr(u) = \frac{2^n}{n+1}$.
  \end{enumerate}
\end{exercise}
\begin{solution}
\end{solution}
\end{enonce}
