\begin{enonce}
\begin{exercise}[ID={RMS135 E954},subtitle={Mines-Ponts PSI 2024},tags={oraux},difficulty={}]
  Soit $\Gamma : x\mapsto \int_0^{+\infty} t^{x-1} e^{-t} \d t$.
  \begin{enumerate}
    \item Montrer que $\Gamma$ est définie sur $]0,+\infty[$ et qu'elle est de classe $\class^2$.
      Montrer de plus que $\Gamma(x)>0$ pour tout $x>0$.

    \item Étudier la convexité de $\Gamma$ et celle de $\ln\circ \Gamma$.


    \item Pour tout $x>0$, établir
      \begin{equation*}
        \lim_{n\to+\infty} \int_0^n t^{x-1} \left( 1 - t/n \right)^n \d t = \Gamma(x).
      \end{equation*}

    \item Exprimer $\int_0^n t^{x-1} \left( 1-t/n \right)^n \d t$ en fonction de $\int_0^1 u^{x-1} \left( 1-u \right)^n \d u$.

    \item Montrer que la suite de fonctions
      \begin{equation*}
        f_n : x\in{\R>}\mapsto \frac{n^x n!}{x(x+1)\dots(x+n)}
      \end{equation*}
      converge simplement vers $\Gamma$.
      \begin{hint}
      Procéder par intégration par parties successives.
      \end{hint}
  \end{enumerate}
\end{exercise}
\begin{solution}
\end{solution}
\end{enonce}
