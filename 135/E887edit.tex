\begin{enonce}
\begin{exercise}[ID={RMS135 E887edit},subtitle={Mines-Ponts PSI 2024},tags={oraux},difficulty={}]
  Soit $C\in\mat_n(\C)$ une matrice de rang $r$.

  \begin{enumerate}
  % \item Démontrer le théorème du rang pour les endomorphismes de $\C^n$.

  \item Montrer qu'il existe $P,Q\in\gl_n(\R)$ telles que $C = P J_r Q$ où $J_r = \begin{pmatrix} I_r & 0\\ 0 & 0 \end{pmatrix}$.

  %%% La questions originale semble manger trop de temps de préparation. Modifié selon le conseil d'Isabelle:
  \item[] On rappelle  qu'il existe $P,Q\in\gl_n(\R)$ telles que $C = P J_r Q$ où $J_r = \begin{pmatrix} I_r & 0\\ 0 & 0 \end{pmatrix}$.

    \item\label{q:RMS135E887c} Soient $A,B\in\mat_n(\C)$ telles que $AC = CB$.
      Montrer que $A$ et $B$ possèdent $r$ valeurs propres communes en tenant compte des multiplicités.

    \item Que peut-on dire dans la question~\ref{q:RMS135E887c} quand $r=n$?
  \end{enumerate}
\end{exercise}
\begin{solution}
\end{solution}
\end{enonce}
