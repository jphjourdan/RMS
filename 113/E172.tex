\begin{enonce}
\begin{exercise}[ID={RMS 113/4 E172},subtitle={Mines-Ponts},tags={},difficulty={}]
%%% 2002
Soient $A\in\mat_n(\C)$ diagonalisable et $P\in\C[X]$ avec $\deg P\geq 1$.
\begin{enumerate}
\item Montrer qu'il existe $M\in\mat_n(\C)$ telle que $A=P(M)$.
\item On suppose de plus que les valeurs propres de $A$ sont simples.
    
    Trouver toutes les matrices $M\in\mat_n(\C)$ telle que $A=P(M)$.
\item Soit $M\in\mat_n(\C)$ dont toutes les valeurs propres sont simples. On suppose que $A$ et $M$ sont co-diagonalisables. 
    
    Établir l'existence de $Q\in\C[X]$ tel que $A=Q(M)$.
\end{enumerate}
\end{exercise}
\begin{solution}
\begin{enumerate}
\item Commençons par noter que la fonction $z\mapsto P(z)$ est surjective de $\mathbb{C}$ dans $\mathbb{C}$.
  En effet, le théorème de d'Alembert-Gauß assure que pour $w\in\mathbb{C}$, le polynôme $P(X)-w$ a au moins une racine dans $\mathbb{C}$.

La matrice $A$ est diagonalisable, il existe donc $\Omega\in\gl_n(\mathbb{C})$ telle que
\begin{equation*}
    \Omega^{-1} A\Omega = \diag(\alpha_1,\dots,\alpha_n)=
\begin{pmatrix}
    \alpha_1 &&\gros 0\\
     &\ddots &\\
    \gros 0&&\alpha_n
\end{pmatrix}.
\end{equation*}
Choisissons pour chaque $k$, $w_k$ tel que $P(w_k)=\alpha_k$. Alors
\begin{gather*}
M
=
\Omega 
\begin{pmatrix}
    w_1 &&\gros 0\\
     &\ddots &\\
    \gros 0&& w_n
\end{pmatrix}
\Omega^{-1}
\shortintertext{vérifie}
P(M)
=
\Omega
\begin{pmatrix}
    P(w_1) &&\gros 0\\
     &\ddots &\\
     \gros 0&& P(w_n)
\end{pmatrix}
\Omega^{-1}
=
A.
\end{gather*}
\item
Supposons $\alpha_1,\dots,\alpha_n$ deux à deux distincts.
Si $A=P(M)$, $M$ commute à $A$. Les sous-espaces propres de $A$, qui sont des droites, sont stables par $M$. Par conséquent, $M$ est nécessairement diagonalisable dans la même base  que $A$ et
\begin{equation*}
\Omega^{-1} M\Omega
=
\begin{pmatrix}
    w_1 &&\gros 0\\
     &\ddots &\\
    \gros 0&& w_n
\end{pmatrix}
\end{equation*}
où, pour chaque $k$, $P(w_k)=\alpha_k$. Il y a un nombre fini de solutions, au plus $N^n$, où $N=\deg P$.
\item
Supposons qu'il existe $\Omega\in\gl_n(\mathbb{C})$ telle que
\begin{align*}
\Omega^{-1}M\Omega&=
\begin{pmatrix}
    w_1 &&\gros 0\\
     &\ddots &\\
    \gros 0&& w_n
\end{pmatrix}
&\text{où $w_1,\dots,w_n$ sont distincts,}\\
\Omega^{-1}A\Omega&=
\begin{pmatrix}
    \alpha_1 &&\gros 0\\
     &\ddots &\\
    \gros 0&& \alpha_n
\end{pmatrix}
&\text{où $\alpha_1,\dots,\alpha_n$ sont quelconques.}
\end{align*}
Il existe un polynôme $Q$ de degré $\leq n-1$, tel que pour tout $k$, $Q(w_k)=\alpha_k$.
C'est le polynôme d'interpolation de Lagrange. Ce polynôme vérifie $Q(M)=A$.

Tous les polynômes congrus à $Q$ modulo le polynôme minimal de $M$, c'est-à-dire $\prod(X-w_i)$, répondent à la question.
\end{enumerate}
\end{solution}
\end{enonce}
