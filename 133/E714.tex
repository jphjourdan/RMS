\begin{enonce}
\begin{exercise}[ID={RMS133 E714},subtitle={Mines-Ponts PSI 2022},tags={},difficulty={}]
  \begin{enumerate}
    \item Étudier la convergence et calculer la somme éventuelle de la série de terme général $u_n$ définie par
      \begin{equation*}
        u_0 = 1 \et \forall n\in\N, u_{n+1} = \ln\left( e^{u_n} - u_n \right).
      \end{equation*}

    \item Étudier la convergence et calculer la somme éventuelle de la série de terme général $v_n$ définie par
      \begin{equation*}
        v_0 \in\R \et \forall n\in\N, v_{n+1} = e^{v_n} - 1.
      \end{equation*}
  \end{enumerate}
\end{exercise}
\begin{solution}
  \begin{enumerate}
    \item Grace à l'inégalité $e^x \geq x+1 > x$, on voit que la suite $(u_n)$ est bien définie, et par récurrence que $u_n > 0$, pour tout $n$.
      On a aussi
      \begin{equation*}
        e^{u_{n+1}} - e^{u_n} = -u_n < 0.
      \end{equation*}
      La suite $u$ est donc décroissante; elle est minorée par $0$ et donc converge.
      Par passage à la limite dans l'égalité supra, cette limite est nulle.
      En outre,
      \begin{equation*}
        \sum_{k=0}^{n} u_k 
        = \sum_{k=0}^{n} \left( e^{u_k} - e^{u_{k+}} \right) 
        = e^{u_0} - e^{u_{n+1}}
        \xto{} e - 1.
      \end{equation*}

      \item 
      On va traiter trois cos:
      \begin{itemize}
        \item Si $v_0=0$, alors $v_n = 0$ pour tout $n$...

        \item Si $v_0>0$, la suite $(v_n)$ est croissante et ne peut converger vers $0$ : la série diverge grossièrement.

        \item Si $v_0<0$, la suit $(v_n)$ est croissante et converge vers $0$ (...).
          De plus, 
          \begin{equation*}
            v_{n+1} = e^{v_n} - 1 = v_n  + \frac12 v_n^2 + o(v_n^2),
          \end{equation*}
          et donc
          \begin{equation*}
            \frac{1}{v_{n+1}} = \frac{1}{v_n}\frac{1}{1 + \frac{1}{2}v_n + o(v_n)} = \frac{1}{v_n} - \frac12 + o(1)
            \implies
            \frac{1}{v_n} - \frac{1}{v_{n+1}} \sim \frac12.
          \end{equation*}
          La série $\sum 1/v_n - 1/{v_{n+1}}$ diverge et 
          \begin{equation*}
            \frac{1}{v_0} - \frac{1}{v_n} \sim \frac{n}{2}
            \implies
            v_n \sim \frac{2}{n}\leq 0.
          \end{equation*}
          La série de terme général $v_n$ est donc divergente.
      \end{itemize}
  \end{enumerate}
\end{solution}
\end{enonce}
