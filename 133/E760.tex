\begin{enonce}
\begin{exercise}[ID={RMS133 E760},subtitle={Mines-Ponts 2022},tags={},difficulty={}]
  Soit $\zeta:s\mapsto \sum_{n=1}^{+\infty} \frac{1}{n^s}$.
  Soit $P$ une probabilité sur $\N*$ définie par
  \begin{equation*}
    \forall n\in\N*, P\left( \set{n} \right) = \frac{c}{n^s}.
  \end{equation*}
  \begin{enumerate}
    \item Déterminer les valeurs de $s$ pour lesquelles il existe une telle probabilité.
      Préciser la valeur de $c$ dans ce cas.

    \item Soit $\sP$ l'ensemble des nombres premiers.
      Pour $p\in\sP$, soit $\Lambda_p$ l'ensemble des entiers naturels non nuls multiples de $p$.
      Montrer que les $\left( \Lambda_p \right)_{p\in\sP}$ sont mutuellement indépendants.
  \end{enumerate}
\end{exercise}
\begin{solution}
\end{solution}
\end{enonce}
