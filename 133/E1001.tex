\begin{enonce}
\begin{exercise}[ID={RMS133 E1001},subtitle={Centrale MP 2022},tags={},difficulty={}]
Un endomorphisme $f$ d'un espace vectoriel $E$ de dimension finie est dit semi-simple si tout sous-espace de $E$ stable par $f$ admet un supplémentaire stable par $f$.
\begin{enumerate}
  \item Montrer que, si $E$ est un $\C$-espace vectoriel de dimension finie, alors les endomorphismes semi-simples de $E$ sont les endomorphismes diagonalisables de $E$.

  \item Si $\K=\R$, montrer que les endomorphismes semi-simple de $E$ sont ceux dont le polynôme minimal n'est divisible par aucun carré non constant de $\R[X]$.
\end{enumerate}
\end{exercise}
\begin{solution}
\end{solution}
\end{enonce}
