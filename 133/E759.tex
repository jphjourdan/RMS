\begin{enonce}
\begin{exercise}[ID={RMS133 E759},subtitle={Mines-Ponts PSI 2022},tags={},difficulty={}]
  Notons, pour $n\in\N*$, $D_n$ le nombre de permutations de $\ito{n}$ sans point fixe et $p_n$ la probabilité qu'une permutation de $\ito{n}$ choisie au hasard soit sans point fixe.
  Par convention, $p_0=1$.

  \begin{enumerate}
    \item Montrer que, pour $n\in\N$,
      \begin{equation*}
        \sum_{k=0}^n \frac{p_k}{(n-k)!} = 1.
      \end{equation*}

    \item En déduire que, pour $x\in]-1,1[$,
      \begin{equation*}
        \sum_{n=0}^{+\infty} p_n x^n = \frac{e^{-x}}{1-x}.
      \end{equation*}

    \item Montrer que $p_n \xto[n\to+\infty]{} 1/e$.
  \end{enumerate}
\end{exercise}
\begin{solution}
  \begin{enumerate}
    \item Il y a $\binom nk D_k$ permutation avec exactement $k$ points fixes, donc
      \begin{equation*}
        \sum_{k=0}^n \binom{n}{k} D_k 
        = \sum_{k=0}^n \frac{n!}{(n-k)!} \frac{D_k}{k!}
        = \sum_{k=0}^n \frac{n!}{(n-k)!} p_k
        = n!.
      \end{equation*}
      D'où le résultat après simplification par $n!$.


    \item On a (après justification des convergence),
      \begin{equation*}
        \sum_{n=0}^{+\infty} x^n
        =
        \sum_{n=0}^{+\infty} \sum_{k=0}^n \frac{x^{n-k}}{(n-k)!} p_k x^k
        =
        \sum_{n=0}^{+\infty} \frac{x^{n}}{n!}  \sum_{n=0}^{+\infty} p_n x^n.
      \end{equation*}
      On a donc
      \begin{equation*}
        \frac{1}{1-x}
        =
        e^x  \sum_{n=0}^{+\infty} p_n x^n.
      \end{equation*}

    \item On a alors
    \begin{equation*}
        \sum_{n=0}^{+\infty} p_n x^n.
        = \frac{e^{-x}}{1-x}
        = \sum_{n=0}^{\infty} \frac{(-1)^n x^n}{n!} \sum_{n=0}^{\infty} x^n
        = \sum_{n=0}^{\infty} \sum_{k=0}^{n} \frac{(-1)^k}{k!}x^k x^{n-k}
        = \sum_{n=0}^{\infty} \left(\sum_{k=0}^{n} \frac{(-1)^k}{k!}\right) x^{n}.
      \end{equation*}
      D'où (unicité des coefficients),
      \begin{equation*}
        p_n = \sum_{k=0}^{n} \frac{(-1)^k}{k!} \xto[n\to+\infty]{} e^{-1}.
      \end{equation*}

  \end{enumerate}
\end{solution}
\end{enonce}
