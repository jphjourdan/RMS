\begin{enonce}
\begin{exercise}[ID={RMS 121-2 E1004},subtitle={},tags={}, difficulty={0}]
  Soient $E_0$ l'espace des $f\in\CC^0(\R,\C)$, $2\pi$-périodiques et $E_1$ l'espace des $f\in\CC^1(\R,\C)$ $2\pi$-périodiques.
  Si $f\in E_1$, soit
  \begin{equation*}
    \Phi(f):x\mapsto \int_0^\pi\frac{f(x+t)-f(x-t)}{t}\d t.
  \end{equation*}
  \begin{enumerate}
    \item Montrer que $\Phi$ est dans $\lin(E_1,E_0)$.
    \item On note $c_n(g)$ les coefficients de Fourier d'une fonction $g$ de $E_0$.
      Si $f\in E_1$, exprimer $c_n\left( \Phi(f) \right)$ en fonction de $c_n(f)$ et de $\alpha_n=\int_0^{n\pi}\frac{\sin t}{t} \d t$.
    \item Montrer qu'il existe $K>0$ tel que
      \begin{equation*}
        \forall f\in E_1, \norm*{\Phi(f)}_2 \leq K\norm*{f}_2.
      \end{equation*}
      Que peut-on en déduire sur $\Phi$ ?
    \item L'application $\Phi$ est-elle injective ? surjective ?
  \end{enumerate}
\end{exercise}
\begin{solution}
\end{solution}
\end{enonce}
