\begin{enonce}
\begin{exercise}[ID={RMS134 E1512},subtitle={IMT MP 2023},theme={analyse},concours={mines-telecom},annee={2023},filiere={MP}]
  On considère
  $f:x\mapsto\dint_0^{+\infty}\dfrac{1-\cos(tx)}{t^2}\e^{-t}\d t$.
  \begin{enumerate}
  \item Donner le domaine de définition de $f$.
  \item Montrer que $f$ est de classe $\mathcal{C}^2$ sur $\R$.
  \item Exprimer $f''$.
  \item En déduire des expressions de $f'$ et $f$ avec des fonctions
    usuelles.
  \end{enumerate}
  
  %exo 1512  RMS 134-1
\end{exercise}
\begin{solution}
  \begin{enumerate}
  \item L'intégrale est bien convergente pour tout $x\in\R$, en
    comparant l'intégrande à $\dfrac{2\e^{-t}}t$ lorsque $t\to
    +\infty$ et en montrant que l'intégrande est prolongeable par
    continuité en $0$ à l'aide d'un développement limité du $\cos$.
  \item Le théorème de dérivation des intégrales à paramètre
    s'applique sans difficulté. On obtient
    \[f''(x)=\int_0^{+\infty}\cos(tx)\e^{-t}\d t\]
  \item Deux IPP successive ou bien (plus rapide) un calcul via une
    exponentielle complexe donnent $f''(x)=\dfrac1{1+x^2}$.
  \item Pour une primitive de $\arctan$, faire une IPP de $1\times
    \arctan(t)$ (même technique que pour $\ln$).
  \end{enumerate}
\end{solution}
\end{enonce}