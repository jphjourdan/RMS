\begin{enonce}
\begin{exercise}[ID={RMS134 E1499},subtitle={CCINP MP 2023},theme={analyse},concours={ccinp},annee={2023},filiere={MP}]
  \begin{enumerate}
  \item Étudier la convergence simple de la série entière $\dsum_{n\ge
      1}\sin\left(\dfrac1{\sqrt{n}}\right)x^n$. On note $D$ l'ensemble
    de convergence et $S(x)$ la somme sur $D$. L'application $S$
    est-elle continue sur $D$ ?
  \item Montrer que $\dsum_{n\ge
      2}\left(\sin\dfrac1{\sqrt{n}}-\sin\dfrac1{\sqrt{n-1}}\right)x^n$
    converge normalement sur $[-1,1]$.
  \item En déduire la valeur de $\dlim_{x\to 1^-}(1-x)S(x)$.
  \end{enumerate}
  %exo 1499 RMS 134-1
\end{exercise}
\begin{solution}
  \begin{enumerate}
  \item On a $\sin\left(\dfrac1{\sqrt{n}}\right)\sim n^{-\frac12}$
    quand $n\to +\infty$ et on sait d'après le cours que toute série
    entière de la forme $\sum n^{\alpha}x^n$ admet $1$ comme rayon de
    convergence. La série diverge pour $x=1$ et converge pour $x=-1$
    (critère spécial des séries alternées) de sorte que $D=[-1,1[$. La
    continuité de $S$ sur $]-1,1[$ est assurée par la convergence
    normale sur tout segment de cet intervalle, tandis que la
    continuité en $-1$ résulte du théorème d'Abel radial: $S$ est bien
    continue sur $D$.
  \item Pour $n\ge 2$, on a:
    \[\sup_{x\in [0,1]}\left|\left(\sin\dfrac1{\sqrt{n}}
          -\sin\dfrac1{\sqrt{n-1}}\right)x^n\right|=
      \sin\dfrac1{\sqrt{n-1}}-\sin\dfrac1{\sqrt{n}},\]
    terme général d'une série telescopique convergente, ce qui nous
    donne donc la convergence normale souhaitée.
  \item Pour $x\in ]-1,1[$, on a:
    \begin{align*}
      (1-x)S(x)
      &=\sum_{n=1}^{+\infty}\sin\left(\dfrac1{\sqrt{n}}\right)x^n-
      \sum_{n=1}^{+\infty}\sin\left(\dfrac1{\sqrt{n}}\right)x^{n+1}\\
      &=\sum_{n=1}^{+\infty}\sin\left(\dfrac1{\sqrt{n}}\right)x^n-
        \sum_{n=2}^{+\infty}\sin\left(\dfrac1{\sqrt{n-1}}\right)x^n\\
      &= \sin(1)x^n+\sum_{n=2}^{+\infty}
        \left(\sin\dfrac1{\sqrt{n}}
        -\sin\dfrac1{\sqrt{n-1}}\right)x^n
    \end{align*}
    On reconnait l'expression de la somme de la série de la question
    précédente, dont la convergence normale sur $[-1,1]$ nous garantit la
    continuité en $1$ et nous permet de passer à la limite et
    d'obtenir:
    \[\dlim_{x\to 1^-}(1-x)S(x)=
      \sin(1)+\sum_{n=2}^{+\infty}
        \left(\sin\dfrac1{\sqrt{n}}
          -\sin\dfrac1{\sqrt{n-1}}\right)
        =\sin(1)-\sin(1)=0\]
  \end{enumerate}
\end{solution}
\end{enonce}