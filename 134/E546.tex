\begin{enonce}
\begin{exercise}[ID={RMS134 E546},subtitle={Oral
    Mines-Ponts},theme={algebre},annee={2023},concours={mines-ponts},filiere={MP}]
  Soit $\K=\Q+\sqrt{2}\Q+\sqrt{3}\Q+\sqrt{6}\Q$. Montrer que $\K$ est
  un $\Q$-sous-espace vectoriel de $\R$ et que
  $(1,\sqrt{2},\sqrt{3},\sqrt{6})$ est une base de $\K$.
  %exo 546 RMS 134-1
\end{exercise}
\begin{solution}
  Montrer successivement par l'absurde que $(1,\sqrt 2)$, $(1,\sqrt
  2,\sqrt 3)$ et $(1,\sqrt 2,\sqrt 3, \sqrt  6)$ sont libres ?
  Pour le dernier cas écrire $\sqrt 6=a+b\sqrt 2+c\sqrt 3$, on
  retrouve 2 autres combinaisons du même type en multipliant par
  $\sqrt 2$ ou $\sqrt 3$ et on peut utiliser la liberté de $(1,\sqrt
  2,\sqrt 3)$ pour obtenir des contradictions si $b$ ou $c$ non nul.
  
  remarque: $\K$ est l'extension de corps $\Q(\sqrt 2,\sqrt 3)$, qui
  coincide en fait que avec $\Q[\sqrt 2,\sqrt 3]$, la
  $\Q$-algèbre engendrée par $\sqrt 2,\sqrt 3$.
\end{solution}
\end{enonce}