\begin{enonce}
\begin{exercise}[ID={RMS134 E883},subtitle={Oral Mines-Ponts},annee={2023},concours={mines-ponts},filiere={MP},theme={probabilites}]
  Soit $p\in]0,1[$. Dans un sac contenant $n$ jetons numérotés de $1$ à
 $n$, on tire $S$ jetons où $S$ est une variable aléatoire suivant une
 loi binomiale de paramètre $n$ et $p$. Quelle est la probabilité
 d'obtenir des jetons de numéros consécutifs?
  %exo 883 RMS 134-1
\end{exercise}
\begin{solution}
  En notant $A$ cet événement, la probalité $\P(A|S=k)$ s'évalue en
  considérant comme univers $\Omega_k$ l'ensemble des parties de
  $\llbracket 1,n\rrbracket$ à $k$ éléments et en dénombrant
  l'ensemble des éléments de $\Omega_k$ constitués de nombres
  consécutifs. Pour $k\ge 1$, il y en a exactement $n-k+1$, et
  on trouve \[\P(A|S=k)=\dfrac{n-k+1}{\combi{n}{k}}.\]
  On a le cas particulier $\P(A|S=0)=0$ et la formule des probabilités
  totales donne donc finalement
  \[\P(A)=\dsum_{k=0}^n\P(A|S=k)\P(S=k)=\dsum_{k=1}^n(n-k+1)p^k(1-p)^{n-k}
    =(1-p)^n\dsum_{k=1}^n(n-k+1)r^k,\]
  avec $r=\dfrac{p}{1-p}$. Considérons alors
  $f:x\mapsto\dfrac{x^{n+1}-x}{x-1}=\dsum_{k=1}^nx^k$, dérivable sur
  $]0,1[$, de dérivée:
  \[f'(x)=\dsum_{k=1}^nkx^{k-1}=\dfrac{nx^{n+1}-(n+1)x^n+1}{(x-1)^2}\]
  On a alors:
  \[\dsum_{k=1}^n(n-k+1)r^k=(n+1)f(r)-rf'(r)=\dfrac{r^{n+2}-(n+1)r^2+nr}{(r-1)^2},\]
  ce qui donne un résultat final pas très drôle.
\end{solution}
\end{enonce}