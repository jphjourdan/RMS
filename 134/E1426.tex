\begin{enonce}
\begin{exercise}[ID={RMS134 E1426},subtitle={IMT MP 2023},theme={algebre},concours={mines-telecom},annee={2023},filiere={MP}]
  \begin{enumerate}
  \item Soit $E$ un $\R$-espace vectoriel de dimension finie et $u$ un
    endomorphisme de $E$. On note $\lambda_1,\ldots,\lambda_p$ ses
    valeurs propres distinctes et $P=\dprod_{k=1}^p(X-\lambda_i)$.

    Donner une condition nécessaire et suffisante portant sur $P$ pour
    que $u$ soit diagonalisable et la démontrer.
  \item Soit $f\in\Lin(\R^7)$. Est-il possible d'avoir simultanément
    $Q=(X-1)(X^2+1)$ annulateur de $f$ est $\tr(f)=0$ ?
  \item Soit $g\in\Lin(\R^7)$ tel que $Q(g)=0$. Calculer $\det(g)$.
  \end{enumerate}
  %exo 1426 RMS 134-1
\end{exercise}
\begin{solution}
  \begin{itemize}
    \item Solution à compléter
  \end{itemize}
\end{solution}
\end{enonce}