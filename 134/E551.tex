\begin{enonce}
\begin{exercise}[ID={RMS134 E551},subtitle={Oral Mines-Ponts},theme={algebre},annee={2023},concours={mines-ponts},filiere={MP}]
  \begin{enumerate}
  \item Soient $p\in\N^*$, $a_1,\ldots,a_p\in\R$ non tous nuls et
    $b_1,\ldots,b_p\in\R$ avec $b_1<\cdots<b_p$. Montrer que
    $f_p:x\mapsto\dsum_{i=1}^pa_i\e^{b_ix}$ s'annule au  plus $p-1$
    fois sur $\R$.
  \item Soit $n\in\N^*$. Soient $\alpha_1<\cdots<\alpha_n$ et
    $\beta_1<\cdots<\beta_n$ des réels. Montrer que
    $\det(\e^{\alpha_i\beta_i})_{1\le i,j\le n}>0$.
  \end{enumerate}
% question a) plutôt bien analyse je dirais
  % exo 551 RMS 134-1
\end{exercise}
\begin{solution}
  \begin{enumerate}
    \item Récurrence sur $p$. Pour l'hérédité, on écrit
      $f_{p+1}(x)=\e^{b_{p+1}x}g_p(x)$ de sorte que
      $g_p'(x)=\dsum_{i=1}^{p}a_i(b_i-b_{p+1})\e^{(b_i-b_{p+1})x}$ s'annule
      au plus $p-1$ fois (HR) et la stricte monotonie entre deux
      points d'annulation successifs prouve que $g_p$, donc $f_{p+1}$,
      s'annule au plus $p$ fois.
    \item
      Si ce déterminant est nul, il existe une combinaison linéaire nulle
      des lignes $\sum a_iC_i$ avec les $a_i$ non tous nuls, et la
      fonction $x\mapsto \dsum_{i=1}^na_i\e^{\alpha_i x}$ s'annule au
      moins $n$ fois (en tous les $\beta_j$), absurde.
      
      %pas mieux que ce qui suit pour l'instant
      Pour la stricte positivité, examinons déjà le cas
      $\alpha_i=\beta_i=i$, il s'agit alors d'un déterminant de
      Vandermonde, à savoir
      $V(\e,\e^2,\ldots,\e^n)=\dprod_{i<j}(\e^j-\e^i)>0$. On construit
      maintenant un chemin continue qui permet de se ramener à ce cas:
      considère l'application $\Phi:[0,1]\to\R$ définie par
      \[\Phi(t)=\det\Big(\e^{\big((\alpha_i-i)t+i\big)\big((\beta_j-j)t+j\big)}\Big)\]
      $\Phi$ est continue sur $[0,1]$, $\Phi(0)>0$ et $\Phi(1)$ est
      le déterminant cherché. Pour les mêmes raisons que le cas $t=1$,
      on a $\Phi(t)\neq 0$ pour tout $t\in]0,1[$. Si $\Phi(1)<0$,
      le théorème des valeurs intermédiaires nous fournit $t\in
      ]0,1[$ tel que $\Phi(t)=0$, absurde.
    \end{enumerate}
\end{solution}
\end{enonce}