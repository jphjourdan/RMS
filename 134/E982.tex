\begin{enonce}
\begin{exercise}[ID={RMS134 E982},subtitle={Mines-Ponts PSI 2023},tags={oraux},difficulty={}]
\begin{enumerate}
  \item Dans une urne, on trouve $n$ boules noires et $n$ boules blanches.
    On tire sans remise des boules jusqu'à ce qu'il ne reste plus de boule d'une des deux couleurs.
    Soit $X_n$ la variable aléatoire qui représente le nombre de boules restantes dans l'urne à la fin des tirages.

    Donner la loi de $X_n$.

  \item On dispose maintenant de deux urnes contenant $n$ boules chacune.
    On tire sans remise de manière équiprobable soit dans l'urne $1$, soit dans l'urne $2$ une boule jusqu'à ce qu'une des deux urnes soit vide.
    Soit $Y_n$ le nombre de boules restantes dans l'urne non vidée après l'expérience.

    Donner la loi de $Y_n$.
\end{enumerate}
\end{exercise}
\begin{solution}
\end{solution}
\end{enonce}
