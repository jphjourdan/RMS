\begin{enonce}
\begin{exercise}[ID={RMS134 E1437},subtitle={CCINP MP 2023},theme={algebre},concours={ccinp},annee={2023},filiere={MP}, difficulty={0}]
  On note $E=\R[X]$.
 \begin{enumerate}
 \item Montrer que l'on définit un produit scalaire sur $E$ en posant
   $\langle P,Q\rangle=\dint_0^1P(t)Q(t)\d t$.
 \item Trouver $a$ et $b$ dans $\R$ tel que $\dint_0^1(t^2-at-b)^2\d
   t$ soit minimal:
   \begin{itemize}
   \item en construisant une base orthonormée de $\R_1[X]$
   \item en recherchant $a,b$ tels que $X^2-aX-b$ soit orthogonal à
     $\R_1[X]$.
   \end{itemize}
 \end{enumerate}
  %exo 1437 RMS 134-1
\end{exercise}
\begin{solution}
  \begin{enumerate}
  \item Facile, ne pas oublier d'invoquer la continuité des polynômes
    pour le caractère défini positif.
  \item
    \begin{itemize}
    \item On applique le procédé d'orthonormalisation de Schmidt à
      $(1,X)$, on trouve $\left(1,2\sqrt{3}(X^2-\dfrac12)\right)$. Le
      min cherché est réalisé pour le le projeté orthogonal de $X^2$
      sur $\R_1[X]$, on trouve $X-\dfrac16$.
    \item Il s'agit encore d'obtenir ce projeté mais sans passer par
      l'expression dans une base orthonormale.
    \item Remarque: on peut aussi trouver $a,b$ en faisant un détour
      par le calcul différentiel et en cherchant les
      points critiques de la fonction $(x,y)\mapsto
      \dint_0^1(t^2-xt-y)^2\d t$.
  \end{itemize}
    
  \end{enumerate}
\end{solution}
\end{enonce}