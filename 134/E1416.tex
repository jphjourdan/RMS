\begin{enonce}
\begin{exercise}[ID={RMS134 E1416},subtitle={IMT MP
                2023},concours={mines-telecom},annee={2023},theme={algebre},filiere={MP}, difficulty={0}]
  Soit $A\in\M_n(\C)$ et $B=A^3+A+I_n$.
  \begin{enumerate}
  \item On suppose que $A$ est diagonalisable, à valeurs propres
    réelles. Montrer que $A$ est un polynôme en $B$.
  \item Est-ce encore vrai si les valeurs propres de $A$ sont
    complexes ?
  \end{enumerate}
  %exo 1416 RMS 134-1
\end{exercise}
\begin{solution}
  \begin{enumerate}
  \item Interpolation de Lagrange
  \item On peut prendre $A\in\M_3(\C)$ dont les coefficients diagonaux
    sont les racines du polynôme $X^3+X+1$, de sorte que $B=0$.
    Hugo propose sinon la matrice $
    A=\begin{pmatrix}
      i & 0\\
      0 & -i
    \end{pmatrix}$, qui donne $B=I_2$.
  \end{enumerate}
\end{solution}
\end{enonce}