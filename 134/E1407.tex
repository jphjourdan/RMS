\begin{enonce}
\begin{exercise}[ID={RMS134 E1407},subtitle={IMT MP
    2023},theme={algebre},concours={mines-telecom},annee={2023},filiere={MP}]
  On considère $E=\{1,2,\ldots,n\}$, $m=2^n-2$, $F_1,\ldots,F_m$ les
  parties de $E$ non triviales (c'est-à-dire dans
  $\mathcal{F}=\mathcal{P}(E)\setminus \{\emptyset, E\}$).
  \begin{enumerate}
  \item Montrer qu'il existe une unique bijection $g$ de $\mathcal{F}$
    dans $\mathcal{F}$ telle que $\forall F\in\mathcal{F},\;g(F)\cap
    F=\emptyset$.
  \item Soit $A=(a_{i,j})_{1\le i,j\le m}\in\M_m(\R)$ telle que
    $a_{i,j}=1$ si $F_i\cap F_j=\emptyset$, $a_{i,j}=0$
    sinon. Calculer $\det(A)$.
  \end{enumerate}
  % exo 1407 RMS 134-1
  % à vérifier dans le bouquin, j'ai échangé $1$ et $0$ pour les
  % valeurs, bien plus cohérent avec la question précédente.
\end{exercise}
\begin{solution}
  \begin{enumerate}
  \item Soit $g$ une telle bijection. Pour tout $F\in\mathcal{F}$ on a
   $|F|+|g(F)|\le n$ et: % idée de Thomas Louboutin
   \[\sum_{F\in\mathcal{F}}(n-|F|-|g(F)|)=
     mn-\sum_{F\in\mathcal{F}}|F|
     -\sum_{F\in\mathcal{F}}|g(F)|
     = mn-\sum_{F\in\mathcal{F}}|F|
     -\sum_{F\in\mathcal{F}}|F^c|=
     mn-\sum_{F\in\mathcal{F}}|E|=0,\]
   en réorganisant la seconde somme des $|g(F)|$ suivant la bijection
   $F\mapsto g(F)^c$. On dispose donc d'une somme nulle à termes posifs
   et on peut en déduire $|F|+|g(F)|= n$ pour tout $F$, de sorte que
   $g(F)=F^c$.
 \item Une permutation $\sigma\in S_m$ s'identifie dans le contexte de
   l'exercice à une bijection $f$ de $\mathcal{F}$, et on a ainsi
   $a_{\sigma(j),j}=1$ si, et seulement si $f(F_j)\cap F_j=\emptyset$. Puisque
   seule la bijection $g$ vérifie cette condition pour tout
   $i\in\llbracket 1,m\rrbracket$, le produit
   $\dprod{j=1}^ma_{\sigma(j),j}$ est nul pour tout
   $\sigma$ autre que la permutation $\sigma_0$ qui s'identifie à $g$,
   et vaut $1$ pour $\sigma=\sigma_0$. On a ainsi:
   \[\det(A)=
     \sum_{\sigma\in
       S_m}\varepsilon(\sigma)\dprod{j=1}^ma_{\sigma(j),j}=
     \varepsilon(\sigma_0)\]
   Remarquons maintenant $\sigma_0$ est involutive (on a $g=g^{-1}$)
   et se décompose comme le produit de $\dfrac{m}2=2^{n-1}-1$
   transpositions à support disjoints. Il s'agit donc d'une
   permutation impaire et on a finalement:
   \[\det(A)=-1\]
  \end{enumerate}
\end{solution}
\end{enonce}