\begin{enonce}
\begin{exercise}[ID={RMS134 E1450},subtitle={IMT MP 2023},difficulty={},tags={algebre, mines-telecom, 2023}]
  \begin{enumerate}[\bfseries a)]
  \item Soit $q\in\N$. On pose $I_q=\dint_O^{+\infty}t^q\e^{-t}\d
    t$. Montrer que $I_q$ est bien définie et que $I_q=q!$.
  \item Pour $P,Q\in\R[X]$, on pose $\langle
    P,Q\rangle=\dint_0^{+\infty}P(t)Q(t)\e^{-t}\d t$. Montrer que
    $\langle\;,\;\rangle$ est un produit scalaire.
  \item Pour $P\in\R_n[X]$, on définit $\phi(P)=XP''+(1-X)P'$. Montrer
    que $\phi$ est un endomorphisme.%de $\R_n[X]$.
  \item Soient $P,Q\in\R[X]$. Montrer que
    $\langle \phi(P),Q\rangle=-\dint_0^{+\infty}tP'(t)Q'(t)\e^{-t}\d
    t$. Montrer que $\phi$ est un endomorphisme symétrique.
  \end{enumerate}
  %exo 1450 RMS 134-1
\end{exercise}
\begin{solution}
  \begin{itemize}
    \item Solution à compléter
  \end{itemize}
\end{solution}
\end{enonce}