\begin{enonce}
\begin{exercise}[ID={RMS134 E561},subtitle={Oral Mines-Ponts},theme={algebre},annee={2023},concours={mines-ponts},filiere={MP}, difficulty={0}]
  Trouver les solutions dans $\M_2(\R)$ de
  $X^2+X=\begin{pmatrix} 1 & 1\\ 1 & 1\end{pmatrix}$.
   %exo 561 RMS 134-1
\end{exercise}
\begin{solution}
  La matrice $M=\begin{pmatrix} 1 & 1\\ 1 & 1\end{pmatrix}$ est
  semblable à $D=\begin{pmatrix} 2 & 0\\ 0 & 0\end{pmatrix}$:
  $M=PDP^{-1}$. En notant $X=PYP^{-1}$, l'équation se ramène à
  $Y^2+Y=D$ dont les seules solutions diagonales sont
  $\begin{pmatrix} 1 & 0\\ 0 & 0\end{pmatrix}$,
  $\begin{pmatrix} 1 & 0\\ 0 & -1\end{pmatrix}$,
  $\begin{pmatrix} -2 & 0\\ 0 & 0\end{pmatrix}$,
  $\begin{pmatrix} -2 & 0\\ 0 & -1\end{pmatrix}$.
  On peut voir que si $\begin{pmatrix} a & b\\ c & d\end{pmatrix}$ est
  une solution non diagonale, alors $a+d=-1$, ce qui conduit (écrire
  le système) à $2=a^2-d^2+a-d=0$, contradiction. Il n'y a
  donc bien que ces quatre solutions, qui correspondent à (une fois
  repassé à $X$)
  $\dfrac12\begin{pmatrix} 1 & 1\\ 1 & 1\end{pmatrix}$,
  $-\begin{pmatrix} 1 & 1\\ 1 & 1\end{pmatrix}$,
  $\begin{pmatrix} 0 & 1\\ 1 & 0\end{pmatrix}$,
  $-\dfrac12\begin{pmatrix} 3 & 1\\ 1 & 3\end{pmatrix}$.
\end{solution}
\end{enonce}