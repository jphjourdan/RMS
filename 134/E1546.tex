\begin{enonce}
\begin{exercise}[ID={RMS134 E1546},subtitle={CCINP MP
    2023},concours={ccinp},annee={2023},filiere={MP}]
  On considère $n$ tulipes qui ont chaque année chacune une probabilité
$p\in]0,1[$ de fleurir, sachant que si une tulipe fleurit une année,
elle fleurira toutes les années suivantes. La variable $X_i$ désigne
l'année de la première floraison de la tulipe numéro $i$, $X$ l'année à
partir de laquelle toutes les tulipes fleurissent.
\begin{enumerate}
\item Exprimer $X$ en fonction des $(X_i)_{i\le n}$.
\item Exprimer la loi des $(X_i)_{i\le n}$.
\item Calculer, pour $k\in\N$, $\P(X>k)$. Montrer que $X$ est
  d'espérance finie et calculer cette espérance.
\end{enumerate}
  % en vrai mines-telecom mais bon, ça ressemble ....
  % exo 1546 RMS 134-1
\end{exercise}
\begin{solution}
  \begin{itemize}
    \item Solution à compléter
  \end{itemize}
\end{solution}
\end{enonce}