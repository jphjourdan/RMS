\begin{enonce}
\begin{exercise}[ID={RMS134 E565},subtitle={Oral Mines-Ponts},theme={algebre},annee={2023},concours={mines-ponts},filiere={MP}]
  Soient $A,B\in\M_n(\R)$. Montrer que $\Ker(A)=\Ker(B)$ si et seulement
s'il existe $P$ inversible tel que $B=PA$.
   %exo 565 RMS 134-1
\end{exercise}
\begin{solution}
  Réciproque claire car $PAX=0\eq AX=0$ en multipliant par
  $P^{-1}$. Supposons $\Ker(A)=\Ker(B)$ et raisonnons avec les
  endomorphismes canoniquement associés $a$ et $b$. Notons $H$ un
  supplémentaire de $\Ker(a)=\Ker(b)$, de sorte que $a$ et $b$
  induisent des isomorphismes $\alpha:H\to\Im(a)$ et
  $\beta:H\to\Im(b)$. On note alors
  $\gamma=\beta\circ\alpha^{-1}:\Im(a)\to\Im(b)$, de sorte que
  $\beta=\gamma\circ\alpha$. $\gamma$ se prolonge en un automorphisme
  $f$ de $\K^n$ (il suffit de compléter arbitrairement des bases de
  $\Im(a)$ et $\Im(b)$) et on a $b=f\circ a$. En notant $P$ la matrice
  de $f$ dans la base canonique, on a bien $B=PA$.
\end{solution}
\end{enonce}