\begin{enonce}
\begin{exercise}[ID={RMS134 E1389},subtitle={CCINP MP 2023},theme={algebre},concours={ccinp},annee={2023},filiere={MP}]
  Soient $A=
 \begin{pmatrix}
   1 & 2 & 3\\
   1 & 0 & 1\\
   1 & 1 & 1\\
   3 & 2 & 1
 \end{pmatrix}$ et $B=
 \begin{pmatrix}
   1\\ 2\\ 3\\ 4
 \end{pmatrix}$.
 \begin{enumerate}
 \item Quel est le rang de $A$ ? Donner une base de l'image de $A$.
 \item Donner une équation de l'image de $A$. Le vecteur $B$
   appartient-il à l'image de $A$ ?
 \end{enumerate}
  %exo 1389 RMS 134-1
\end{exercise}
\begin{solution}
  \begin{enumerate}
  \item Le rang vaut $3$, les trois colonnes donnent une base de
    l'image.
  \item L'image est un hyperplan, noyau d'une forme linéaire,
    représentée par une matrice ligne $L$ telle que $LA=0$, ce qui
    revient à résoudre $A^TX=0$ (où $X=L^T$). Un pivot de gauss donne
    l'équation $x_1-4x_3+x_4=0$. $B$ n'est pas dans l'image.
  \end{enumerate}
\end{solution}
\end{enonce}