\begin{enonce}
\begin{exercise}[ID={RMS134 E550},subtitle={Oral Mines-Ponts},difficulty={},tags={algebre, mines-ponts, 2023}]
  Soit $A=
\begin{pmatrix}
  1 & 1 & 0 & \cdots & 0\\
  0 & \ddots & \ddots & \ddots & \vdots\\
  \vdots & \ddots & \ddots & \ddots & 0\\
  \vdots & & \ddots & \ddots & 1\\
  0 & \cdots & \cdots & 0 & 1
\end{pmatrix}$ et $N=A-I_n$. Soit $(E)$ l'équation matricielle $X^2=A$.
\begin{enumerate}[\bfseries a)]
\item Quelles sont les matrices qui commutent avec $N$ ?
\item Montrer que les solutions de $(E)$ sont de la forme
  $X=\pm
  \begin{pmatrix}
    1 & a_1 & \cdots & a_{n-1}\\
    0 & \ddots & \ddots & \vdots\\
    \vdots & \ddots & \ddots & a_1\\
    0 & \cdots & 0 & 1
  \end{pmatrix}$.

  Montrer qu'il y a au plus deux solutions.
\item Rappeler le développement limité à l'ordre $n$ de
  $x\mapsto\sqrt{1+x}$. Résoudre $(E)$.
\end{enumerate}
   %exo 550 RMS 134-1
\end{exercise}
\begin{solution}
  \begin{enumerate}[\bfseries a)]
  \item Pour $B=(b_{i,j})$ on a $[BN]_{i,j}=b_{i,j-1}$ ($0$ si $j=1$) et
    $[NB]_{i,j}=b_{i+1,j}$ ($0$ si $i=n$). $BN=NB$ implique donc d'une
    part que $B$ est constante sur ses diagonales, et d'autre part
    (par une récurrence qu'il faudrait mettre en place) qu'elle est
    triangulaire supérieure. Les matrices qui commutent avec $N$ sont
    donc de la forme
    $\begin{pmatrix}
    a_0 & a_1 & \cdots & a_{n-1}\\
    0 & \ddots & \ddots & \vdots\\
    \vdots & \ddots & \ddots & a_1\\
    0 & \cdots & 0 & a_0
  \end{pmatrix}$.
  \item Si $B$ est solution de $(E)$, on a $B^2=N+I_n$ donc
    $BN+B=B^3=NB+B$ ce qui prouve déjà que $B$ est de la forme
    précédente. Mais on doit avoir de plus $a_0^2=1$ et on a donc
    $a_0=\pm 1$. Quitte à changer $a_i en -a_i$ pour $i\ge 1$ dans le
    cas $a_0=-1$, on a bien $B$ de la forme voulue. Notons maintenant
    que pour $i<j$, on a, en notant $a_0=1$:
    \[[B^2]_{i,j}=\dsum_{k=1}^nb_{i,k}b_{k,j}=\dsum_{k=i}^ja_{k-i}a_{j-k}
      = 2a_{j-i}+\dsum_{k=i+1}^{j-1}a_{k-i}a_{j-k}\]
    On en déduit $a_1=\dfrac12$ et pour tout $p\ge 2$ (on a posé
    $p=j-i$ et décalé les indices)
    \[a_p=-\dfrac12\dsum_{k=1}^{p-1}a_ka_{p-k}\]
    Cela prouve par récurrence que les $a_i$ sont uniques et qu'il y a
    donc au plus deux solutions pour l'équation $(E)$.
  \item On retrouve (péniblement ?):
    \[\sqrt{1+x}=\sum_{k=0}^n\dfrac{(-1)^{k+1}}{(2k-1)2^{2k}}\combi{2k}{k}x^k
      +\o(x^n)\]
   Inspiré par cette écriture, il reste à justfier correctement qu'en
   posant
   \[B=\sum_{k=0}^n\dfrac{(-1)^{k+1}}{(2k-1)2^{2k}}\combi{2k}{k}N^k,\]
   c'est-à-dire $B=\begin{pmatrix}
     a_0 & a_1 & \cdots & a_{n-1}\\
     0 & \ddots & \ddots & \vdots\\
     \vdots & \ddots & \ddots & a_1\\
     0 & \cdots & 0 & a_0
   \end{pmatrix}$, avec
   $a_k=\dfrac{(-1)^{k+1}}{(2k-1)2^{2k}}\combi{2k}{k}$ (on retrouve
   notamment $a_0=1$ et $a_1=\dfrac12$), on a $B^2=I_n+N=A$, et donc que
   $\pm B$ sont les deux solutions de $(E)$.   
  \end{enumerate}
\end{solution}
\end{enonce}