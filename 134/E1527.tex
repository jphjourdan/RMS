\begin{enonce}
\begin{exercise}[ID={RMS134 E1527},subtitle={CCINP MP 2023},difficulty={},tags={analyse, ccinp, 2023}]
  \begin{enumerate}[\bfseries a)]
  \item Déterminer les extrema de $f:(u,v)\in[0,1]^2\mapsto
    uv(1-u-v)$.
  \item Soit $(A,B,C)$ un triangle d'aire égale à $1$. Soit $M$ un
    point dans le triangle. Maximiser le produit des distances de $M$
    aux côtés du triangle.
  \end{enumerate}
  %exo 1527RMS 134-1
\end{exercise}
\begin{solution}
  \begin{enumerate}[\bfseries a)]
  \item Déterminons déjà les points critiques de $f$. Cette fonction
    est de classe $\mathcal{C}^{\infty}$ par théorèmes d'opérations et
    on trouve
    \[\dfrac{\partial f}{\partial
        u}(u,v)=v(1-2u-v)\qquad\text{et}\qquad
      \dfrac{\partial f}{\partial
        v}(u,v)=u(1-2v-u)\]
    En étendant le domaine de définition à $\R^2$, on montre sans
    difficulté que les deux dérivées partielles s'anulent en $(u,v)$
    ssi $(u,v)\in\{(0,0), (1,0), (0,1), (\frac13,\frac13)\}$. Une
    étude de la Hessienne en chacun de ces points peut nous fournir la
    nature de ces points critiques. On a de façon générale
    \[H_f(u,v)=
      \begin{pmatrix}
        -2v & 1-2u-2v\\
        1-2u-2v & -2u
      \end{pmatrix}\]
    On constate que $H_f(0,0)$, $H_f(1,0)$ et $H_f(0,1)$ ont toutes
    les trois un déterminant qui vaut $-1$, ce qui correspond donc à
    des points de type selle. En revanche on a:
    \[H_f(\frac13,\frac13)=-\frac13
      \begin{pmatrix}
        2 & 1\\
        1 & 2
      \end{pmatrix},\]
    de déterminant $1>0$ et de trace $-\frac43<0$ ce qui correspond à
    un maximum local. Ce maximum est même global sur $[0,1]^2$
    puisqu'il n'y a pas d'autre point critique dans $]0,1[^2$ et
    puisque $f$ est nulle ou strictement négative sur la frontière de
    $[0,1]^2$.

    {\itshape Remarque: on peut aussi invoquer le théorème des bornes
      atteintes qui assure de l'existence d'un maximum et d'un minimum
      globaux pour $f$ sur le compact $[0,1]^2$. L'étude des
      Hessiennes n'est alors pas nécessaire, le maximum ne pouvant
      être atteint qu'en l'unique point critique dans $]0,1[^2$, car
      $f$ y est alors supérieure à ses valeurs sur la frontière. On
      voit facilement qu'un minimum global est atteint en $(1,1)$. Vu la
      question suivante, il semblerait peut-être
      plus naturel d'étudier les extrema de $f$ sur le triangle $T=OIJ$
      avec $O=(0,0)$, $I=(1,0)$ et $J=(0,1)$. Dans ces conditions, $f$
      est nulle sur tout la frontière, strictement positive à
      l'intérieur et admet donc toujours cet unique maximum en
      $(\frac13,\frac13)$ tandis que toute les points de la frontière
      peuvent être considérés comme des minima locaux.}
  \item Notons $x,y,z$ les distances respectives aux droites $(BC)$,
    $(AC)$ et $(AB)$. Un petit dessin montre clairement que les
    triangles $MBC$, $MAC$ et $MAB$ ont pour aires respectives
    \[u=\dfrac{ax}2,\qquad v=\dfrac{by}{2},\qquad w=\dfrac{cz}2,\]
    avec $a,b,c$ les longueurs respectives des côtés $[BC]$, $[AC]$ et
    $[AB]$. On voit ainsi que le produit $xyz$ des distances est
    maximal ssi le produit $uvw$ est maximal. Or la somme des aires
    des triangles $MBC$, $MAC$ et $MAB$ est égale à l'aire du triangle
    $ABC$ de sorte que $u+v+w=1$, de sorte que
    $uvw=uv(1-u-v)=f(u,v)$. Le produit $xyz$ est donc maximal lorsque
    $u=v=w=\dfrac13$, d'après la question précédente, et vaut alors
    $\dfrac{8}{27abc}$. Le point $M$ doit en tout cas être placé de
    telle sorte que les segment $[MA]$, $[MB]$ et $[MC]$ découpent le
    triangle $ABC$ en trois triangles d'aires égales.
  \end{enumerate}
\end{solution}
\end{enonce}