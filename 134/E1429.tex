\begin{enonce}
\begin{exercise}[ID={RMS134 E1429},subtitle={CCINP MP 2023},theme={algebre},concours={ccinp},annee={2023},filiere={MP}, difficulty={0}]
  Soit $n\in\N^*$ et $N_n$ l'ensemble des matrices nilpotentes de
 $\M_n(\C)$. Soit $A\in\M_n(\C)$. Montrer que les propositions
 suivantes sont équivalentes:

 i) $A$ est diagonalisable;\qquad\qquad
 ii) $\forall P\in\C[X],\quad P(A)\in N_n\eq P(A)=0$.
  %exo 1429 RMS 134-1
\end{exercise}
\begin{solution}
  \begin{itemize}
  \item Supposons $i)$, avec $A$ semblable à
    $\mathrm{diag}(\lambda_1,\ldots,\lambda_n)$, et soit $P\in\C[X]$ de sorte
    que $P(A)$ est semblable à
    $\mathrm{diag}(P(\lambda_1),\ldots,P(\lambda_n))$. Supposons
    $(P(A))^q=0$ pour un certain $q\N$. On a alors $P(\lambda_i)^q=0$,
    et donc $P(\lambda_i)=0$ pour tout $i$, d'où $P(A)=0$. 
  \item Supposons $ii)$, et notons
    $P=(X-\lambda_1)\cdots(X-\lambda_q)$ avec
    $\lambda_1,\ldots,\lambda_r$ les valeurs propres $2$ à $2$
    distinctes. En notant $q$ la plus grande multiplicité des
    $\lambda_i$ dans le polynôme minimal (ou dans le polynôme
    caractéristique, ou même simplement $q=n$), on aura nécessairement
    $(P(A))^q=0$, et donc $P(A)=0$. 
  \end{itemize}
\end{solution}
\end{enonce}