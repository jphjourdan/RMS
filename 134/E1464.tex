\begin{enonce}
\begin{exercise}[ID={RMS134 E1464},subtitle={IMT MP 2023},theme={analyse},concours={mines-telecom},annee={2023},filiere={MP}]
  Soient $E$ un espace euclidien et $\mathcal{K}$ l'ensemble des
  projecteurs orthogonaux de $E$. Soit $p$ un projecteur.
  \begin{enumerate}
  \item Montrer que: $p\in\mathcal{K} \eq \forall x\in E,
    \|p(x)\|\le \|x\|$.
  \item Montrer que $\mathcal{K}$ est un compact.
  \end{enumerate}
  %exo 1464  RMS 134-1
\end{exercise}
\begin{solution}
  \begin{enumerate}
  \item Classique. Si $p$ est un projecteur orthogonal, $\|p(x)\|^2\le
    \|p(x)\|^2+(x-\|p(x)\|)^2=\|x\|^2$ puisque $p(x)$ et $x-p(x)$ sont
    orthogonaux. Pour la réciproque, on fixe $x\in \Ker(p)$ et
    $y\in\Im(p)$ et on écrit que l'hypotèse donne pour tout $t\in\R$:
    \[t^2\|y\|^2=\|p(x+ty)\|^2\le \|x+ty\|^2=\|x\|^2+2t\langle
      x,y\rangle+t^2\|y\|^2\]
    On exhibe ainsi une fonction affine $t\mapsto \|x\|^2+2t\langle
      x,y\rangle$ positive, ce qui n'est possible qu'avec $\langle
      x,y\rangle=0$.
    \item Munissons $\Lin(E)$ de la norme d'opérateur
      $\vvvert\;\vvvert$. La question précédente prouve que
      $\vvvert p\vvvert\le 1$ pour tout $p\in\mathcal{K}$ (on a même
      en fait $\vvvert p\vvvert=1$ pour $p$ non nul), et donc que
      $\mathcal{K}$ est une partie bornée. Comme $\Lin(E)$ est de
      dimension finie, il suffit pour conclure de montrer que
      $\mathcal{K}$ est fermée et on procède par caractérisation
      séquentielle. Soit $(p_n)_n$ une suite de $\mathcal{K}$ qui
      converge vers $p\in\Lin(E)$. Alors:
      \begin{itemize}
      \item $p_n\circ p_n=p_n$ pour tout $n\in\N$ et donc $p\circ p=p$
        en passant à la limite. $p$ est donc un projecteur.
      \item Pour tout $x\in E$, $\|p_n(x)\|\le \|x\|$ pour tout
        $n\in\N$ d'après la question précédente, et donc $\|p(x)\|\le
        \|x\|$ en passant à la limite. D'où $p\in\mathcal{K}$ d'après
        la question précédente.
      \end{itemize}
  \end{enumerate}
\end{solution}
\end{enonce}