\begin{enonce}
\begin{exercise}[ID={RMS134 E1493},subtitle={CCINP MP 2023},theme={analyse},concours={ccinp},annee={2023},filiere={MP}, difficulty={0}]
  Soit $(a_n)_{n\ge 0}$ une suite décroissante de réels positifs qui
  converge vers $0$. Pour tout $t\in[0,1]$, on pose
  $u_n(t)=a_n(1-t)t^n$.
  \begin{enumerate}
  \item Montrer que la série de fonctions $\dsum u_u$ converge
    simplement sur $[0,1]$.
  \item Trouver une condition nécessaire et suffisante pour que cette
    série converge normalement.
  \item Montrer que la série $\sum u_n$ converge uniformément sur
    $[0,1]$.
  \end{enumerate}
  %exo 1493 RMS 134-1
\end{exercise}
\begin{solution}
  \begin{enumerate}
  \item Comparaison série géométrique si $t<1$.
  \item le $\sup|(1-t)t^n|$ est atteint en $\dfrac{n}{n+1}$, et on
    trouve $\|u_n\|_{\infty}\sim \dfrac{a_n}{en}$. La CNS est donc
    $\sum\dfrac{u_n}{n}$ converge.
  \item Pour $\varepsilon>0$ fixé, on dispose de $n_0$ tel que $0\le
    a_n\le \varepsilon$ si $n\ge n_0$. On a alors, pour $n\ge n_0$:
    \[|R_n(t)|\le \varepsilon\left|\sum_{k\ge n+1}(1-t)t^k\right|
      \le \varepsilon t^{n+1}\le \varepsilon\]
  \end{enumerate}
\end{solution}
\end{enonce}