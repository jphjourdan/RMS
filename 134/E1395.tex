\begin{enonce}
\begin{exercise}[ID={RMS134 E1395},subtitle={IMT MP 2023},difficulty={},tags={algebre, mines-telecom, 2023}]
  Soient $A,B\in\M_n(\C)$. On suppose que $A$ est inversible, que $B$
  est nilpotente et que $A$ et $B$ commutent.
  \begin{enumerate}[\bfseries a)]
  \item Montrer que $A-B$ et $A+B$ sont inversibles.
  \item Si $A$ et $B$ ne commutent pas, montrer qu'alors $A+B$ n'est
    pas forcément inversible.
  \end{enumerate}
  %exo 1395 RMS 134-1
\end{exercise}
\begin{solution}
  \begin{enumerate}[\bfseries a)]
  \item Posons $M=A^{-1}B$ de sorte que $A-B=A(I_n-M)$. $A^{-1}$ et
    $B$ commutent (facile), donc $M$ est nilpotente. $I_n-M$ est donc
    inversible d'inverse $\dsum_{k=0}^{m-1}M^k$, avec $m$ le
    nilindice, et il en résulte bien que $A-B$ est inversible, avec
    \[(A-B)^{-1}=(I_n-M)^{-1}A^{-1}=\sum_{k=0}^{m-1}A^{-k-1}B^k\]
    On a bien sûr immédiatement l'inversibilité de $A+B$ également, en
    remplaçant $B$ par $-B$.
  \item Il suffit de considérer par exemple $A=
    \begin{pmatrix}
      0 & -1\\
      1 & 0
    \end{pmatrix}$ et $B=
    \begin{pmatrix}
      0 & 1\\
      0 & 0
    \end{pmatrix}$.
  \end{enumerate}
\end{solution}
\end{enonce}