\begin{enonce}
\begin{exercise}[ID={RMS134 E699},subtitle={Mines-Ponts MP 2023},theme={analyse},annee={2023},concours={mines-ponts},filiere={MP}, difficulty={0}]
  Soit $(b_n)_{n\in\N}$ une suite strictement positive, croissante, et
 non majorée.
 \begin{enumerate}
 \item Montrer que, si $(a_n)_{n\in\N}$ est une suite réelle
   convergente de limite $\ell$, alors
   \[\dfrac1{n_n}\sum_{k=0}^{n-1}(b_{k+1}-b_k)a_k
     \underset{n\to +\infty}\longrightarrow\ell\]
 \item La réciproque de la propriété précédente est-elle vraie ?
 \end{enumerate}
  %exo 699 RMS 134-1
\end{exercise}
\begin{solution}
  \begin{itemize}
    \item Solution à compléter
  \end{itemize}
\end{solution}
\end{enonce}