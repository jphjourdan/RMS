\begin{enonce}
\begin{exercise}[ID={RMS134 E557},subtitle={Oral Mines-Ponts},theme={algebre},annee={2023},concours={mines-ponts},filiere={MP}, difficulty={0}]
  Soit $n\in\N^*$. Calculer $\det((i\wedge j)_{1\le i,j\le n})$.

  Indication: On rappelle que pour $N\in\N^*$,
  $N=\dsum_{d|N}\varphi(d)$ où $\varphi$ est l'indicatrice d'Euler.
   %exo 557 RMS 134-1
\end{exercise}
\begin{solution}
  En notant $M=(i\wedge j)_{1\le i,j\le n})$,
  $D=\mathrm{diag}(\varphi(1),\ldots,\varphi(n)$ et $A=(a_{i,j})$ avec
  $a_{i,j}=1$ si $i|j$ et $0$ sinon, on peut montrer que
  $M=A^TDA$. En effet, $[DA]_{k,j}=\varphi(k)$ si $k|j$, $0$ sinon,
  tandis que $[A^T]_{i,k}=1$ si $k|i$, $0$ sinon, de sorte que dans la
  somme $[A^TDA]_{i,j}=\dsum_{k=1}^n[A^T]_{i,k}[DA]_{k,j}$, le
  $k$-ième terme est $\varphi(k)$ ssi $k|i$ et $k|j$ donc ssi
  $k|(i\wedge j)$, $0$ sinon. On a donc exactement
  $[A^TDA]_{i,j}=i\wedge j$ d'après la formule rappelée. $A$ étant
  triangulaire à coefficients diagonaux $=1$, on a
  $\det(M)=\det(D)=\dprod\varphi(i)$.
\end{solution}
\end{enonce}