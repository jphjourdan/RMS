\begin{enonce}
\begin{exercise}[ID={RMS134 E558},subtitle={Oral Mines-Ponts},difficulty={},tags={algebre, mines-ponts, 2023}]
  Soit $A\in\M_n(\C)$ nilpotente.
\begin{enumerate}[\bfseries a)]
\item Calculer $\det(A+I_n)$.
\item Soit $M\in\M_n(\C)$ telle que $AM=MA$. Calculer $\det(A+M)$. On
  commencera par le cas où $M\in\GL_n(\C)$.
\item Le résultat est-t-il toujours vrai si $AM\neq MA$ ?
\end{enumerate}
   %exo 558 RMS 134-1
\end{exercise}
\begin{solution}
  \begin{enumerate}[\bfseries a)]
  \item $A+I_n$ a pour spectre $\{1\}$ donc
    $\det(A+I_n)=1$.
  \item Si $A\in\GL_n(\C)$, on a $AM^{-1}=M^{-1}A$ et $AM^{-1}$ est
    donc nilpotente, d'où
    $\det(A+M)=\det(AM^{-1}+I_n)\det(M)=\det(M)$. Cette égalité se
    prolonge au cas $M\notin\GL_n(\C)$ par densité de $\GL_n(\C)$ dans
    $\M_n(\C)$ et continuité du déterminant.

    {\itshape N'y a-t-il pas une façon plus directe de montrer que
      $\det(A+M)=0$ lorsque $A$ nilpotent et $M$ non inversible ?}
  \item Contre-exemple avec $A=
    \begin{pmatrix}
      0 & 1 \\ 0 & 0
    \end{pmatrix}$ et $M=
    \begin{pmatrix}
      1 & 0 \\ 1 & 1
    \end{pmatrix}$
  \end{enumerate}
\end{solution}
\end{enonce}