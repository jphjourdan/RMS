\begin{enonce}
\begin{exercise}[ID={RMS134 E556},subtitle={Oral Mines-Ponts},theme={algebre},annee={2023},concours={mines-ponts},filiere={MP}, difficulty={0}]
  Soient $p,q\in\C$. On note $x_1,x_2,x_3$ les racines (non
nécessairement distinctes) du polynôme $X^3+pX+q$. Pour $j\in\N$, on
pose $N_j=x_1^j+x_2^j+x_3^j$.

Calculer, pour $n\in\N^*$, le déterminant de la matrice
$M_n=(N_{i+j-2})_{1\le i,j\le n}$.
   %exo 556 RMS 134-1
\end{exercise}
\begin{solution}
  Les relations coefficients/racines permettent d'obtenir que $N_1=0$,
  $N_2=-2p$ et $N_3=-3q$, de sorte qu'on calcule explicitement
  $\det(M_1)=3$, $\det(M_2)=-6p$, $\det(M_3)=-4p^3-27q^2$ (tiens donc,
  le discriminant ...)
  Puisque $x_i^{j+3}+px_i^{j+1}+qx_i^{j}=0$ pour $1\le i\le 3$ et
  $j\ge 0$, on a $N_{j+3}+pN_{j+1}+N_j=0$ ce qui fournit une
  combinaison linéaire nulle des quatres premières colonnes de $M_n$ dès que
  $n\ge 4$, et donc $\det(M_n)=0$ dans ce cas.

  
  Remarque: Y a-t-il plus simple que de relier cet exercice aux formules
  de Newton ? On a, très 
  généralement, pour $x_1,\ldots x_n$ et $\sigma_1,\ldots,\sigma_n$
  les fonctions symétriques associées, $N_p-\sigma_1
  N_{p-1}+\cdots+(-1)^n\sigma_nN_{p-n}=0$ pour $p\ge n$ (facile en
  passant par les relations coeffs / fonctions symétriques comme dans
  l'exercice) et $N_p-\sigma_1
  N_{p-1}+\cdots+(-1)^{p-1}\sigma_{p-1}N_1+(-1)^pp\sigma_p=0$ pour
  $1\le p\le n-1$ (plus dur).
\end{solution}
\end{enonce}