\begin{enonce}
\begin{exercise}[ID={RMS134 E1501},subtitle={IMT MP 2023},theme={analyse},concours={mines-telecom},annee={2023},filiere={MP}, difficulty={0}]
  On pose $I_n=\dint_0^{+\infty}\dfrac{\sin(nt)}{1+n^4t^3}\d t$ pour
  $n\ge 1$
  \begin{enumerate}
  \item Montrer que $I_n$ est bien définie.
  \item Déterminer $\lim_{n\to +\infty} I_n$.
  \item Nature de la série $\dsum I_n$ ?
  \end{enumerate}
  %exo 1501 RMS 134-1
\end{exercise}
\begin{solution}
  \begin{enumerate}
  \item facile.
  \item convergence dominée, avec la domination $|f_n(t)|\le
    \dfrac1{1+t^3}$.
  \item Une solution qui a l'air de marcher consiste à effectuer le
    découpage $I_n=J_n+K_n$, avec
    $J_n$ l'intégrale de $0$ à $\dfrac1{n}$ et $K_n$ l'intégrale de
    $\dfrac1{n}$ à $+\infty$. Pour la première, on majore $\sin(nt)$
    par $nt$, ce qui donne un majorant pour l'intégrande dont on peut
    étudier les variations. On constate que le max est atteint en
    $x_n=\dfrac{\alpha}{n^{\frac43}}$ (faire le calcul) ce qui donne
    une valeur en $\dfrac1{n^{\frac13}}$ et obtient finalement
    $J_n=\O\left(\dfrac1{n^{\frac43}}\right)$. Pour la seconde on majore plus bêtement
    par $\dfrac1{n^4t^3}$ ce qui s'intégre facilement et donne
    $K_n=\O\left(\dfrac1{n^2}\right)$. La série est donc absolument convergente.
  \end{enumerate}
\end{solution}
\end{enonce}