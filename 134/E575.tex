\begin{enonce}
\begin{exercise}[ID={RMS134 E575},subtitle={Oral Mines-Ponts},difficulty={},tags={algebre, mines-ponts, 2023}]
  Soit $A\in\M_n(\R)$.
  \begin{enumerate}[\bfseries a)]
  \item Montrer que $\|\det
    A\|\le\dprod_{i=1}^n\left(\dsum_{j=1}^n|A_{i,j}|\right)$.
  \item Lorsque $\det A\neq 0$, étudier le cas d'égalité.
  \end{enumerate}
   %exo 575 RMS 134-1
\end{exercice}

\begin{exercice}[concours=Mines-Ponts]
Soit $E$ un $\K$-espace vectoriel. Une partie $S$ de $\Lin(E)$ est
dite dense si, pour tout $n\ge 1$, toute famille $(b_1,\ldots,b_n)$ de
vecteurs de $E$ et toute famille libre $(a_1,\ldots,a_n)$ de vecteurs
de $E$, il existe $f\in S$ tel que $f(a_i)=b_i$ pour tout
$i\in\llbracket 1,n\rrbracket$.
\begin{enumerate}[\bfseries a)]
\item Quelles sont les parties denses de $\Lin(E)$ si $E$ est de
  dimension finie ?
\item Dans cette question, on suppose que $E$ n'est pas de dimension
  finie.
  \begin{enumerate}[(i)]
  \item Montrer que $\{f\in\Lin(E),\; \rg(f)< +\infty\}$ est dense dans
    $\Lin(E)$.
  \item Même question avec $\{f\in\Lin(E),\; \rg(f)\text{ fini et
      pair}\}$.
  \item Si $S$ est dense dans $\Lin(E)$, déterminer
    $\{g\in\Lin(E)\,;\,\forall f\in S, fg=gf\}$.
  \end{enumerate}
\end{enumerate}
   %exo 576 RMS 134-1
\end{exercise}
\begin{solution}
  \begin{enumerate}[\bfseries a)]
  \item Il n'y a que $\Lin(E)$. En effet, une base $(e_1,\ldots,e_n)$
    étant fixée, tout $g\in\Lin(E)$ coincide avec $f\in S$ qui
    vérifierait $f(e_i)=g(e_i)$.
    \begin{enumerate}[(i)]
    \item Remarque: il faut admettre que tout sous-espace de dimension
      finie admet un supplémentaire. En considérant un supplémentaire
      $H$ de $\vect(a_1,\ldots,a_n)$, il suffit de définir $f$ par
      $f(a_i)=b_i$ et $f(x)=0$ si $x\in H$. On a alors $\rg(f)\le n$.
    \item Comme précédemment si $\rg(b_1,\ldots,b_n)$ est pair, sinon
      on se ramène à ce cas en fixant arbitrairement
      $a_{n_1}\notin\vect(a_1,\ldots,a_n)$ et
      $b_{n+1}\notin\vect(b_1,\ldots,b_n)$.
    \item Soit $g\in\Lin(E)$ tel que $fg=gf$ pour tout $f\in
      S$. Fixons $x\in E$ non nul et notons $y=g(x)$. Supposons
      $(x,y)$ libre. Il existe alors $f\in S$ tel que
      $f(x)=f(y)=x$. On a alors $y=g(f(x))=f(g(x))=x$, absurde. Il
      existe donc $\lambda\in\K$ tel que $g(x)=\lambda x$. Ceci est
      vrai pour tout $x\in E$ et on en déduit (exercice classique de
      montrer que $\lambda$ ne dépend pas de $x$) que $g=\lambda id_E$.
    \end{enumerate}
  \end{enumerate}
\end{solution}
\end{enonce}