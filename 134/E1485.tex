\begin{enonce}
\begin{exercise}[ID={RMS134 E1485},subtitle={IMT MP 2023},theme={analyse},concours={mines-telecom},annee={2023},filiere={MP}]
  On considère la suite $(u_n)_{n\ge 0}$ de fonctions définies sur $\R$
 par $u_0=id$ et, pour $n\in\N$, $u_{n+1}=\sin\circ u_n+u_n$.
 \begin{enumerate}
 \item Étudier la convergence simple de $(u_n)$.
 \item La convergence est-elle uniforme ?
 \end{enumerate}
  %exo 1485 RMS 134-1
\end{exercise}
\begin{solution}
  \begin{enumerate}
  \item Pour $x\in\R$ fixé, la suite $(u_n(x))$ s'écrit plus
    simplement $(v_n)$, avec $v_0=x$ et $v_{n+1}=f(v_n)$ avec
    $f:t\mapsto\sin(t)+t$. $f$ est croissante et admet $\pi\Z$ comme
    ensemble de points fixes. $(v_n)$ est donc monotone et doit
    converger vers un multiple de $\pi$. Le sens de monotonie dépend
    du signe de $v_1-v_0$, et on a $v_0\le v_1$ ssi $\sin(x)\le
    0$. Pour $x\in ]n\pi,(n+1)\pi[$, on différentie alors les cas $n$
    pairs et impairs. On obtient que la limite $u$ de $u_n$ est la
    fonction constante par morceaux $x\mapsto (2n+1)\pi$ si
    $x\in]2n\pi,(2n+2)\pi[$, et $2n\pi\mapsto 2n\pi$. 
  \item Non, bien sûr, la limite n'étant pas continue. Il doit y avoir
    convergence sur les segment $[a,b]$ si $2n\pi<a<b<(2n+2)\pi$ pour
    un certain $n\in\Z$.
  \end{enumerate}
\end{solution}
\end{enonce}