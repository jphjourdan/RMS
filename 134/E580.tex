\begin{enonce}
\begin{exercise}[ID={RMS134 E580},subtitle={Oral Mines-Ponts},difficulty={},tags={algebre, mines-ponts, 2023}]
  Soit $M\in\M_n(\C)$ admettant $n$ valeurs propres distinctes. Montrer
que l'ensemble des matrices qui commutent avec $M$ est
$\vect(I_n,M,\ldots,M^{n-1})$.
   %exo 580 RMS 134-1
\end{exercise}
\begin{solution}
  On peut commencer avec
  $M=D=\mathrm{diag}(\lambda_1,\ldots,\lambda_n)$. Si $AD=DA$, on a
  $\lambda_ja_{i,j}=[AD]_{i,j}=[DA]_{i,j}=\lambda_ia_{i,j}$ et donc
  $a_{i,j}=0$ si $i\neq j$. Le commutant de $D$ est donc $D_n(\K)$,
  sev des matrices diagonales, de dimension $n$. Il contient
  $\vect(I_n,D,\ldots,D^{n-1})$ et on montre que la famille
  $(I_n,D,\ldots,D^{n-1})$ est libre: une combinaison linéaire nulle
  fournit un polynôme de $\C_{n-1}[X]$ à au moins $n$ racines
  distinctes, donc nul. On a donc
  $D_n(\K)=\vect(I_n,D\ldots,D^{n-1})$ (remarque: La décomposition
  d'une matrice diagonale donnée s'obtient par interpolation de
  Lagrange).

  Dans le cas général, on peut se ramener au cas précédent en
  diagonalisant: on écrit $M=PDP^{-1}$. Si $B=PAP^{-1}$ est dans le
  commutant de $M$, $A$ est dans le commutant de $D$ donc
  $A\in\vect(I_n,D,\ldots,D^{n-1})$ et
  $B\in\vect(I_n,M,\ldots,M^{n-1})$ (remarque: on peut aussi
  facilement affirmer que $(I_n,M,\ldots,M^{n-1})$ est une base de
  $\K[M]$ car le polynôme minimal de $M$ est de degré $n$).
\end{solution}
\end{enonce}