\begin{enonce}
\begin{exercise}[ID={RMS134 E1399},subtitle={IMT MP 2023},theme={algebre},concours={mines-telecom},annee={2023},filiere={MP}]
  \begin{enumerate}
  \item Soient $n\in\N^*$, $u$ et $v$ deux endomorphismes nilpotents
    non nuls de $\R^n$ tels que $u\circ v=v\circ u$. Montrer que
    $\rg(u\circ v)< \rg(v)$.
  \item Montrer que la composée de $n$ endomorphismes nilpotents de
    $\R^n$ qui commutent deux à deux est nulle.
  \end{enumerate}
  %exo 1399 RMS 134-1
\end{exercise}
\begin{solution}
  \begin{enumerate}
  \item On a déjà $\Im(u\circ v)=\Im(v\circ u)\subset\Im(v)$, donc
    $\rg(u\circ v)\le \rg(v)$. Par l'absurde, supposons
    $\rg(u\circ v)=\rg(v)$. On a alors l'égalité
    $\Im(v)=\Im(u\circ v)$ et on montre alors par réccurrence
    $\Im(v)\subset \Im(u^k\circ v)$ pour tout $k\in\N$. La nilpotence
    de $u$ implique $v=0$, contradiction.
  \item Soient $u_1,\ldots,u_n$ de tels endomorphismes. On montre par
    récurrence sur $k\in\llbracket 1,n\rrbracket$ que
    $\rg(u_1\circ\cdots\circ u_k)\le n-k$.
    \begin{itemize}
    \item Le cas $k=1$ résulte de ce que $u_1$ est nilpotent donc non
      inversible.
    \item Si c'est vrai pour un certain $k\in\llbracket 1,n-1\rrbracket$, on
      applique la question précédente avec $u=u_1\circ\cdots\circ u_k$
      et $v=u_{k+1}$: $u$ est nilpotente car les $u_i$ le sont et
      commutent, $u$ et $v$ commutent, et on a alors:
      \[\rg(u_1\circ\cdots\circ u_{k+1})<
        \rg(u_1\circ\cdots\circ u_k)\le n-k,\]
      par hypothèse de réccurrence, d'où $\rg(u_1\circ\cdots\circ
      u_{k+1})\le n-k-1$.
    \end{itemize}
    La propriété est finalement vraie pour tout
    $k\le n$, et le cas $k=n$ donne en particulier
    $u_1\circ\cdots\circ u_n=0$.
  \end{enumerate}
\end{solution}
\end{enonce}