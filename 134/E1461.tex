\begin{enonce}
\begin{exercise}[ID={RMS134 E1461},subtitle={IMT MP 2023},theme={analyse},concours={mines-telecom},annee={2023},filiere={MP}]
  Soit $N$ définie sur $\R[X]$ par
 $N\left(\dsum_{i=0}^na_iX^i\right)=\dmax_{0\le i\le n}|a_i|$.
 \begin{enumerate}
 \item Montrer que $N$ est une norme.
 \item Soient $a\in\R$ et $\phi:P\in\R[X]\mapsto P(a)$. Pour quelles
   valeurs de $a$ l'application $\phi$ est-elle continue pour la
   norme $N$ ?
 \end{enumerate}
  %exo 1461  RMS 134-1
\end{exercise}
\begin{solution}
  \begin{enumerate}
  \item Pas de difficulté.%, la restriction de $N$ à $\R_n[X]$
  %  % correspond en fait à la norme infini de $\R_{n+1}$.
  \item Supposons $|a|<1$. Pour tout $P=\sum_{k=0}^da_kX^k$ non
    nul de degré $d$, on a
    \[\big|\phi(P)\big|=\left|\sum_{k=0}^da_ka^k\right|
      \le\sum_{k=0}^d|a_k|\,|a|^k
    \le N(P)\sum_{k=0}^d|a|^k
    \le N(P)\dfrac{1}{1-|a|}\]
    On en déduit que $\phi$ est continu, par critère fondamental de
    continuité des applications linéaires (il faut bien sûr justifier
    que $\phi$ est linéaire). Supposons maintenant $|a|\ge 1$ et
    considérons la suite de
    polynômes $(P_n)$ définie par
    \[\forall n\in\N,\quad P_n=\dfrac1{n+1}\sum_{k=0}^n\dfrac1{a^k}X^k\]
    On a ainsi $N(P_n)
    =\dfrac1{n+1}\dmax_{0\le i\le n}\dfrac1{|a|^i}=\dfrac1{n+1}\to 0$, donc
    $P_n\to 0$ au sens de la
    norme $N$. Cependant, pour tout $n\in\N$, on a
    \[\phi(P_n)=P_n(a)=\dfrac1{n+1}\sum_{k=0}^n\dfrac{a^k}{a^k}=1,\]
    alors que la continuité de $\phi$ en $0$ devrait impliquer
    $\phi(P_n)\to 0$ par caractérisation séquentielle. On a donc
    montré que $\phi$ est continue si, et seulement si $|a|<1$.
    
    {\itshape Remarque: cet exercice et sa solution sont encore
      valables si on se place dans $\C[X]$.}
  \end{enumerate}
\end{solution}
\end{enonce}