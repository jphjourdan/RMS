\begin{enonce}
\begin{exercise}[ID={RMS134 E554},subtitle={Oral Mines-Ponts},theme={algebre},annee={2023},concours={mines-ponts},filiere={MP}]
  %% coquille dans l'énoncé qui n'imposait pas $a\in\Ker(h)$
  %% remarque: la définition donnée exclu le cas $h=id$, censé
  %% correspondre tout de même à une transvection normalement (voir
  %% wikipedia)
Soit $E$ un $\K$-espace vectoriel de dimension finie. On dit que
$h\in\Lin(E)$ est une transvection s'il existe $\varphi\in\Lin(E,\K)$
non nulle et $a\in \Ker(\varphi)$ non nul tels que: $\forall x\in
E,h(x)=x+\varphi(x)a$. Soit $u\in\Lin(E)$ tel que $\rg(u-id)=1$ et
$(u-id)^2=0$. Montrer que $u$ est une transvection. La réciproque
est-elle vraie?
   %exo 554 RMS 134-1
\end{exercise}
\begin{solution}
  Soit $a\neq 0$ dans $\Im(u-id)$ de sorte qu'on peut écrire
  $(u-id)(x)=\varphi(x)a$ pour tout $x\in E$, ce qui définit une forme
  linéaire $\varphi$ telle que $u(x)=x+\varphi(x)a$ pour tout $x$.
  On a alors $(u-id)^2(x)=(u-id)(\varphi(x)a)=\varphi(x)\varphi(a)a=0$
  pour tout $x$ ce qui prouve $\varphi(a)=0$.

  Réciproquement, soit une transvection $u=id+\varphi(\cdot)a$. On a
  $u-id\subset\vect(a)$ et $u-id\neq 0$ puisque $\varphi\neq 0$, donc
  $\rg(u-id)=1$. Avec de plus $a\in\Ker(\varphi)$, on a comme
  précédemment $(u-id)^2(x)=\varphi(x)\varphi(a)a=0$ pour tout $x$.
\end{solution}
\end{enonce}