\begin{enonce}
\begin{exercise}[ID={RMS134 E1459},subtitle={IMT MP 2023},theme={analyse},concours={mines-telecom},annee={2023},filiere={MP}, difficulty={0}]
  Soit $E$ un $\R$-espace vectoriel normé de dimension finie. Soit
 $(u_n)_{n\in\N}\in E^{\N}$ une suite telle que, pour tout vecteur
 $x\in E$, la suite $(\|x-u_n\|)_{n\in\N}$ converge.
 \begin{enumerate}
 \item Montrer que la suite $(u_n)_{n\in\N}$ a une valeur d'adhérence.
 \item Montrer que la suite $(u_n)_{n\in\N}$ converge.
 \end{enumerate}
  
  %exo 1459  RMS 134-1
\end{exercise}
\begin{solution}
  \begin{enumerate}
  \item Avec $x=0$ on a en particulier la convergence de la suite
    $(\|u_n|_{\infty})$ et la suite $(u_n)$ est donc bornée. $E$ étant
    de dimension finie, le théorème de Bolzano-Weirestrass nous assure
    alors de l'existence d'une valeur d'adhérence $\ell$ pour $(u_n)$.
  \item Par hypothèse, la suite $(\|\ell-u_n\|)_{n\in\N}$ converge
    vers un certain réel $a\ge 0$. Mais $\ell$ étant une valeur
    d'adhérence pour $(u_n)$, cette suite admet une suite extraite
    $(u_{\varphi(n)})$ qui converge vers $\ell$ et la suite extraite
    $(\|\ell-u_{\varphi(n)}\|)$ converge donc vers $0$. Comme elle
    converge aussi vers $a$, on a $a=0$ par unicité de la limite et
    donc $u_n\to \ell$.
  \end{enumerate}
\end{solution}
\end{enonce}