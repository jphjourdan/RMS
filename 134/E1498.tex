\begin{enonce}
\begin{exercise}[ID={RMS134 E1498},subtitle={CCINP MP 2023},theme={analyse},concours={ccinp},annee={2023},filiere={MP}]
  On pose, pour $n\in\N$, $I_n=\dint_{0}^{\frac{\pi}{4}}\tan^{n}(t)\d t$.
  \begin{enumerate}
  \item Montrer, pour $n\in\N$, $0\le I_n\le\dfrac{\pi}{4}$. En
    déduire que le rayon de convergence de $\dsum I_nx^n$ est $\ge 1$.
  \item Montrer, pour $n\in\N$, que $I_{n+2}+I_n=\dfrac1{n+1}$.
  \item Donner un équivalent simple de $I_n$.
  \item Déterminer le rayon de convergence $R$ de $\dsum
    I_nx^n$. Calculer $\dsum_{n=0}^{+\infty}I_nx^n$ pour $x\in
    ]-R,R[$.
  \end{enumerate}
  %exo 1498 RMS 134-1
\end{exercise}
\begin{solution}
  \begin{itemize}
    \item Solution à compléter
  \end{itemize}
\end{solution}
\end{enonce}