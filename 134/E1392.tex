\begin{enonce}
\begin{exercise}[ID={RMS134 E1392},subtitle={IMT MP 2023},theme={algebre},concours={mines-telecom},annee={2023},filiere={MP}, difficulty={0}]
  La matrice $A=
  \begin{pmatrix}
    3 & 2\\ 1 & 2
  \end{pmatrix}\in\M_2(\R)$ est-elle inversible ? Déterminer $A^n$
  pour tout $n\in\Z$.
  %exo 1392 RMS 134-1
  % -1 initialement à la place du 1, mais calculs plus bourrins
\end{exercise}
\begin{solution}
  On a $\det(A)=4$, donc la matrice est inversible. Son polynôme
  caractéristique est $\chi_A=X^2-5X+4=(X-4)(X-1)$. $A$ est donc
  semblable à la matrice $D=\mathrm{diag}(4,1)$, et il existe
  $P\in\GL_2(\R)$ telle que $A=PDP^{-1}$, d'où l'on tire
  $A^n=PD^nP^{-1}$ pour tout $n\in\N$. Pas besoin de chercher $P$ et
  $P^1$, on sait que les coefficients de $A^n$ vont être de la forme
  $a+4^nb$, et il suffit de résoudre $4$ petits systèmes.
  \[A^n=\]
\end{solution}
\end{enonce}