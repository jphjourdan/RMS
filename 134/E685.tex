\begin{enonce}
\begin{exercise}[ID={RMS134 E685},subtitle={Mines-Ponts MP 2023},theme={analyse},annee={2023},concours={mines-ponts},filiere={MP}]
  On munit l'espace $E=\mathcal{C}^0([0,1],\R)$ du produit scalaire
 usuel défini par
 $\langle f,g\rangle = \int_0^1f(t)g(t)\d t$ et de la norme associé
 $\|\;\|_2$. Soit $F$ un sous-espace de $E$ tel qu'il existe une
 constante $C\in\R$ telle que $\forall f\in F,\,\|f\|_{\infty}\le
 C\|f\|_2$.
 \begin{enumerate}
 \item Montrer que $F\neq E$.
 \item Soit $(f_1,\ldots,f_n)$ une famille orthonormale de $F$.

   Montrer que $\forall a_1,\ldots,a_n\in\R$,
   $\left|\dsum_{i=1}^na_if_i\right|\le C\sqrt{\dsum_{i=1}^na_i^2}$.
 \item En déduire que $F$ est de dimension finie majorée par $C^2$.
 \end{enumerate}
  %exo 685 RMS 134-1
\end{exercise}
\begin{solution}
  \begin{itemize}
    \item Solution à compléter
  \end{itemize}
\end{solution}
\end{enonce}