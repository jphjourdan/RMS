\begin{enonce}
\begin{exercise}[ID={RMS134 E581},subtitle={Oral Mines-Ponts},theme={algebre},annee={2023},concours={mines-ponts},filiere={MP}]
  Soient $n\in\N^*$, $E$ un $\R$-espace vectoriel de dimension finie et
$u\in\Lin(E)$ tel que $u^n=id$. Pour $b\in E$ et $\lambda\in\R$,
résoudre $x+\lambda u(x)=b$.
   %exo 581 RMS 134-1
\end{exercise}
\begin{solution}
  Que dire d'autre que $x=(u+\lambda id)^{-1}(b)$ lorsque $-\lambda$
  n'est pas valeur propre, notamment lorsque $\lambda\neq \pm 1$ ?
\end{solution}
\end{enonce}