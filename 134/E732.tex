\begin{enonce}
\begin{exercise}[ID={RMS134 E732},subtitle={Mines-Ponts MP 2023},theme={analyse},annee={2023},concours={mines-ponts},filiere={MP}]
  Pour toute permutation $f\in\N^*$, on note
  $E_f=\left\{\alpha\in\R, \dsum\dfrac{f(n)}{n^\alpha}<+\infty\right\}$.
    \begin{enumerate}
    \item Montrer qu'il existe $f\in\mathcal{S}(\N^*)$ tel que
      $E_f=\emptyset$.
    \item Soit $f\in\mathcal{S}(\N^*)$. Montrer que si
      $E_f\neq\emptyset$, alors c'est un intervalle minoré par $2$ et
      non majoré.
    \item Montrer que, si $\beta>2$, alors il existe
      $f\in\mathcal{S}(\N^*)$ tel que $E_f=]\beta,+\infty[$.
    \end{enumerate}
  %exo 732 RMS 134-1
\end{exercise}
\begin{solution}
  \begin{enumerate}
  \item Notons $A=\{2^n, n\in\N^*\}$ et $B=\N^*\setminus
    A=\{u_n,n\in\N^*\}$ ($B$ étant bien sûr dénombrable). On pose
    $f(2n)=2^n$ et $f(2n-1)=u_n$ pour tout $n\in\N^*$, ce qui définit
    bien $f\in\mathcal{S}(\N^*)$. Pour $\alpha\in\R$, on a
    $\dsum\dfrac{f(n)}{n^\alpha}>\dsum_{n=1}^{+\infty}\dfrac{2^n}{(2n)^{\alpha}}=+\infty$.
  \item Il suffit de montrer que $\dsum\dfrac{f(n)}{n^2}=+\infty$. En
    effet, on aura alors $\alpha<2\implique \alpha\notin E_f$ par
    comparaison des séries positives, et $\alpha\in E_f\implique
    [\alpha,+\infty[\subset E_F$, toujours par comparaison. Pour
    $n\in\N$, on a nécessairement:
    \[\sum_{k=2^n+1}^{2^{n+1}}f(k)\ge \sum_{k=1}^{2^n}k\ge 2^{2n-1}\]
    On en déduit la minoration:
    \[\sum_{k=2^n+1}^{2^{n+1}}\dfrac{f(k)}{k^2}\ge
      \dfrac{2^{2n-1}}{2^{2n}}=\dfrac12\]
    Il en résulte facilement le résultat annoncé par sommation par
    critère de Cauchy ou sommation par paquets.
  \item Pas encore cherché ...
  \end{enumerate}
\end{solution}
\end{enonce}