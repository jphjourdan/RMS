\begin{enonce}
\begin{exercise}[ID={RMS134 E949},subtitle={Oral Mines-Pont PSI
    2023},tags={oraux,mpsi}, concours=mines-ponts, annee=2023,
  filiere=PSI, theme=analyse]
Soient $E$ un $\R$-espace vectoriel, $N_1$ et $N_2$ deux normes sur $E$.
\begin{enumerate}
  \item Soit $(u_n)$ une suite qui converge dans $(E,N_1)$.
    On suppose que $N_1$ et $N_2$ sont équivalentes.
    Montrer que $(u_n)$ converge dans $(E,N_2)$.

  \item On suppose qu'une suite $(u_n)$ converge dans $(E,N_1)$ si, et seulement si $(u_n)$ converge dans $(E,N_2)$.
    Montrer que $N_1$ et $N_2$ sont équivalentes.

  \item On prend $E=\R[X]$, $a\in\R$, et
    \begin{equation*}
      N_a(P) = \abs{P(a)} + \int_0^1 \abs*{P'(t)}\d t.
    \end{equation*}
    Montrer que si $a,b\in[0,1]$, $N_a$ et $N_b$ sont équivalentes.

  \item Soit, pour $n\in\N$, $P_n = \frac{X^n}{2^n}$.
    Trouver les valeurs de $a$ telles que $(P_n)$ converge pour $N_a$ et déterminer alors la limite.

  \item En déduire que $N_a$ et $N_b$ ne sont pas équivalentes si $0\leq a < b$ et $b>1$.
\end{enumerate}
\end{exercise}
\begin{solution}
\end{solution}
\end{enonce}
