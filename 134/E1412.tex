\begin{enonce}
\begin{exercise}[ID={RMS134 E1412},subtitle={IMT MP 2023},difficulty={},tags={algebre, mines-telecom, 2023}]
  Soit $E$ un $\C$-espace vectoriel, $u\in\Lin(E)$ diagonalisable,
  $e=(e_1,\ldots,e_n)$ une base de vecteurs propres.
  \begin{enumerate}[\bfseries a)]
  \item Montrer que $\chi_u(u)=0$ sans utiliser le théorème de
    Cayley-Hamilton.
  \item On écrit $x=\dsum_{i=1}^nx_ie_i$. Calculer
    $\det_e\big(x,u(x),\ldots,u^{n-1}(x)\big)$.
  \end{enumerate}
  %exo 1412 RMS 134-1
\end{exercise}
\begin{solution}
  \begin{enumerate}[\bfseries a)]
  \item Dans la base $e$, $u$ est représentée par une matrice
    diagonale $\diag(\lambda_1,\ldots,\lambda_n)$, de sorte que
    $\chi_u(u)$ est représentée par
    $\diag(\chi_u(\lambda_1),\ldots,\chi_u(\lambda_n))$.
  \item les composantes de $u^k(x)$ dans $e$ sont
    $(x_1\lambda_1^k,\ldots, x_n\lambda_n^k)$. On a donc
    \[
      \Mat{e}\big(x,u(x),\ldots,u^{n-1}(x)\big)
      =
      \begin{pmatrix}
        x_1 & \ldots & x_1\lambda_1^{n-1}\\
        \vdots & \ddots & \vdots\\
        x_n & \ldots & x_n\lambda_n^{n-1}
      \end{pmatrix}
    \]
    La linéarité ligne par ligne du déterminant permet de le
    factoriser ici par le produit des $x_i$, et apparait alors un
    déterminant de Vandermonde:
    \[\det_e\big(x,u(x),\ldots,u^{n-1}(x)\big)=
      \prod_{i=1}^nx_i
      \begin{vmatrix}
        1 & & \ldots & \lambda_1^{n-1}\\
        \vdots & \ddots & \vdots\\
        1 & \ldots & \lambda_n^{n-1}
      \end{vmatrix}=
      \prod_{i=1}^nx_i\prod_{1\le i<j\le n}(\lambda_j-\lambda_i)\]
  \end{enumerate}
\end{solution}
\end{enonce}