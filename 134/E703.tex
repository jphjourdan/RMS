\begin{enonce}
\begin{exercise}[ID={RMS134 E703},subtitle={Mines-Ponts MP 2023},theme={analyse},annee={2023},concours={mines-ponts},filiere={MP}, difficulty={0}]
  Pour tout $n\ge 2$, on pose $f_n(x)=x^n-nx+1$.
 \begin{enumerate}
 \item Montrer que l'équation $f_n(x)=0$ admet une unique solution
   $x_n$ dans $[0,1]$.
 \item Étudier la monotonie de la suite $(x_n)$. Montrer sa
   convergence.
 \item Déterminer la limite de la suite $(x_n)$ et un équivalent
   simple de $x_n$.
 \item Déterminer un développement asymptotique à deux termes de
   $x_n$.
 \end{enumerate}
  %exo 703 RMS 134-1
\end{exercise}
\begin{solution}
  \begin{enumerate}
  \item $f_n'(x)=n(x^{n-1}-1)$ donne la décroissance de $f_n$ sur
    $[0,1]$. Avec $f(0)=1$ et $f(1)=2-n$, le TVI assure l'existence et
    l'unicité de $x_n\in[0,1]$ tel que $f_n(x_n)=0$.
  \item En utilisant $x_n^n-nx_n+1=0$, on a
    \[f_{n+1}(x_n)=x_n^{n+1}-(n+1)x_n+1=x_n^{n}(x_n-1)-x_n\le 0,\]
    et la décroissance de $f_{n+1}$ donne $x_{n+1}\le x_n$, puisque
    $f_{n+1}(x_{n+1})=0$.
  \item On a $0\le nx_n=x_n^n+1\le 2$, d'où $x_n\to 0$. À partir d'un
    certain rang, $x_n\le \dfrac12$, et donc
    \[0\le nx_n-1=x_n^n\le \dfrac1{2^n}\to 0\]
    On en déduit $nx_n\to 1$ et donc $x_n\sim\dfrac1{n}$.
  \item On peut affiner ce qui précéde et écrire qu'à partir d'un
    certain rang, on a:
    \[0\le n^2x_n-n\le \dfrac{n}{2^n}\to 0,\text{ par croissance
        comparée.}\]
    Il en résulte $nx_n-1=\o\left(\dfrac1{n}\right)$, ce qui permet
    d'écrire:
    \[0\le
      nx_n-1=\dfrac1{n^n}(nx_n)^n=\dfrac1{n^n}\left(1+\o\left(\dfrac1{n}\right)\right)^n\]
    Or, on a:
    \[\left(1+\o\left(\dfrac1{n}\right)\right)^n=
      \exp\left[n\ln\left(1+\o\left(\dfrac1{n}\right)\right)\right]
      =\exp\big(\o(1)\big)\to 1\]
    On en déduit que $nx_n-1\sim\dfrac1{n^n}$ et on a finalement:
    \[x_n=\dfrac1{n}+\dfrac1{n^{n+1}}+\o\left(\dfrac1{n^{n+1}}\right)\]
  \end{enumerate}
\end{solution}
\end{enonce}