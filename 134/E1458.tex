\begin{enonce}
\begin{exercise}[ID={RMS134 E1458},subtitle={IMT MP 2023},theme={analyse},concours={mines-telecom},annee={2023},filiere={MP}]
  On note $E$ l'ensemble des fonctions $f\in\mathcal{C}^1([0,1],\R)$
 telles que $f(0)=0$. Pour $f\in E$, on pose $N(f)=\|f+f'\|_{\infty}$
 et $N'(f)=\|f\|_{\infty}+\|f'\|_{\infty}$.
 \begin{enumerate}
 \item Montrer que $N$ et $N'$ sont des normes sur $E$.
 \item Montrer que $N$ et $N'$ sont équivalentes.

   {\itshape Indication: exprimer $f$ en fonction de $g=f+f'$}
 \end{enumerate}
  
  %exo 1458  RMS 134-1
\end{exercise}
\begin{solution}
  \begin{enumerate}
  \item Pas de difficulté pour l'inégalité triangulaire et
    l'homogénéité. Pour la séparation de $N$, dire que $N(f+f')=0$
    implique que $f$ est solution sur $[0,1]$ du problème de Cauchy
    défini par $y'=-y$ et $y(0)=0$ qui admet la fonction nulle comme
    unique solution.
  \item On a clairement $N\le N'$. Pour l'autre sens, posons
    $g=f+f'$ comme indiqué. $f$ est solution de l'équation
    différentielle linéaire $y+y'=g$, de sorte qu'en posant
    $k(x)=\e^{x}f(x)$, on a $k'(x)=\e^xg(x)$ et donc:
    \[f'(x)=g(x)-k(x)\e^{-x}=g(x)-\e^{-x}\int_0^xg(t)\e^t\d t\]
    On en déduit:
    \[|f'(x)|\le |g(x)|+\int_0^x\|g\|_{\infty}\e^t\d t
      \le \|g\|_{\infty}\left(1+\int_0^x\e^t\d t\right)\le
      e\|g\|_{\infty}\]
    On a donc, par passage à la borne supérieure, $\|f'\|_{\infty}\le
    \e\|g\|_{\infty}$.

    Par ailleurs, l'inégalité des accroissements finis prouve
    $|f(x)|=|f(x)-f(0)|\le x\|f\|_{\infty}$ pour tout $x\in [0,1]$ et
    donc $\|f\|_{\infty}\le \|f'\|_{\infty}$. On a donc finalement
    
    \[N'(f)\le 2\|f'\|_{\infty}\le 2\e N(f)\]
  \end{enumerate}
\end{solution}
\end{enonce}