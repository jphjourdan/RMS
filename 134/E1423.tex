\begin{enonce}
\begin{exercise}[ID={RMS134 E1423},subtitle={CCINP MP 2023},concours={ccinp},annee={2023},theme={algebre},filiere={MP}, difficulty={0}]
  Soient $A\in\M_n(\R)$ et $B=
 \begin{pmatrix}
   A & A\\ 0 & A
 \end{pmatrix}$.
 \begin{enumerate}
 \item Montrer que, pour $P\in\R[X]$, $P(B)=
   \begin{pmatrix}
     P(A) & AP'(A)\\
     0 & P(A)
   \end{pmatrix}$.
 \item Donner une condition nécessaire et suffisante sur $A$ pour que
   $B$ soit diagonalisable.
 \end{enumerate}
  %exo 1423 RMS 134-1
\end{exercise}
\begin{solution}
  \begin{enumerate}
  \item Une récurrence permet de montrer que pour tout $k\in\N$, on a
    $B^k= \begin{pmatrix}
     A^k & kA^k\\
     0 & kA^k
   \end{pmatrix}$. La propriété à montrer est donc vraie pour tous les
   polynômes de la base canonique $(X^k)$. On en déduit facilement,
   par linéarité, qu'elle est vraie pour tout polynôme $P$.
 \item Supposons $B$ diagonalisable. Il existe donc un polynôme $P$
   scindé à racines simples et annulateur de $B$. La question
   précédente permet d'affirmer qu'on alors $P(A)=0$ et $AP'(A)=0$. On
   en déduit que $A$ est diagonalisable et que toute valeur propre de
   $A$ est racine de $P$. Notons maintenant que $P'$ est scindé sur
   $\C$ et peut s'écrire:
   \[P'=\alpha\prod_{i=1}^d(X-\lambda_i),\]
   avec $\lambda_1,\ldots,\lambda_d\in\C$. Le point crucial est
   qu'aucun des $\lambda_i$ n'est racine de $P$, puisque $P$ est
   scindé à racines simples. Il en résulte que $A-\lambda_iI_n$ est
   inversible pour tout $i\in\llbracket 1,d\rrbracket$, et donc que
   $P'(A)$ est inversible. On en déduit immédiatement la condition
   nécessaire $A=0$, qui est bien sûr suffisante pour que $B$ soit
   diagonalisable.
  \end{enumerate}
\end{solution}
\end{enonce}