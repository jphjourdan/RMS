\begin{enonce}
  \begin{exercise}[ID={RMS134 E958},subtitle={Mines-Ponts PSI 2023},tags={oraux},difficulty={}]
  % Analyse
  Soit $E=\class^\infty(\R,\R)$.
  Soit $p\in\left]-1,1\right[\setminus\set{0}$ et $q\in\R$ tel que $p+q=1$.
  Soit $f\in E$.
  On définit $u$ comme l'application de $E$ dans $E$ qui à $f$ associe la fonction $g:x\mapsto f(px+q)$.
  \begin{enumerate}
    \item Montrer que $u$ est un automorphisme.
    \item Montrer que toutes les valeurs propres de $u$ sont dans $\left]-1,1\right]$.
    \item Montrer que si $f$ est vecteur propre de $u$, il existe un entier $k$ tel que $f^{(k)}=0$.
      En déduire l'ensemble des vecteurs propres de $u$.

    \item Calculer $u^n(f)(x)$ par récurrence. %% ??
  \end{enumerate}
  \end{exercise}
  \begin{solution}
  \end{solution}
\end{enonce}
