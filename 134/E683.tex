\begin{enonce}
\begin{exercise}[ID={RMS134 E683},subtitle={Mines-Ponts MP 2023},theme={analyse},annee={2023},concours={mines-ponts},filiere={MP}]
  \begin{enumerate}
  \item Soit $P\in\R[X]$ unitaire de degré $n\in\N^*$. Montrer que $P$
    est scindé sur $\R$ si et seulement si: $\forall
    z\in\C,\,|P(z)|\ge|\Im(z)|^n$.
  \item Montrer que l'adhérence des matrices de $\M_n(\R)$
    trigonalisables est fermé.
  \item Quelle est l'adhérence de l'ensemble des matrices
    diagonalisables de $\M_n(\R)$?
  \end{enumerate}
  %exo 683 RMS 134-1
\end{exercise}
\begin{solution}
  \begin{enumerate}
  \item Si $P=\dprod_{k=1}^n(X-\lambda_k)$, on a $|z-\lambda_k|\ge
    |\Im(z-\lambda_k)|=|\Im(z)|$ pour tout $k$, d'où $|P(z)|\ge
    |\Im(z)|^n$. Réciproquement, si on a cette inégalité pour tout
    $z$, on a $\Im(z)=0$ pour toute racine complexe, et $P$ est donc
    scindé sur $\R$.
  \item Soit $(M_k)_k$ est une suite convergente de matrices trigonalisables sur
    $\R$, de limite $M$, et $(\chi_k)_k$ la suite des polynômes
    caractéristiques. Si $z\in\C$, on a $|\chi_k(z)|\ge |\Im(z)|^n$
    pour tout $k$, et $\|\chi_M(z)|\ge |\Im(z)|^n$ en passant à la
    limite (continuité de $A\mapsto \chi_A$ et de $P\mapsto P(z)$).
  \item Cette adhérence est déjà contenue dans l'ensemble fermé des
    matrices trigonalisable d'après la question précédente. Pour
    l'inclusion réciproque, on peut fixer une matrice trigonalisable
    $M=PTP^{-1}$. On peut montrer alors que
    $T_k=T+\mathrm{diag}\left(\dfrac{i}{k}\right)_{1\le i\le n}$ est
    diagonalisable à partir d'un certain rang (valeurs propres $2$ à
    $2$ distinctes). On a donc $M_k=PT_kP^{-1}$ diagonalisable et
    $M_k\to M$ par continuité du produit matriciel.
  \end{enumerate}
\end{solution}
\end{enonce}