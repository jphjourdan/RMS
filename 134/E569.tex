\begin{enonce}
\begin{exercise}[ID={RMS134 E569},subtitle={Oral Mines-Ponts},difficulty={},tags={algebre, mines-ponts, 2023}]
  Soient $E$ un espace vectoriel de dimension finie et
$(u,v)\in\Lin(E)^2$.
\begin{enumerate}[\bfseries a)]
\item Montrer que
  $\big|\rg(u)-\rg(v)\big|\le\rg(u+v)\le\rg(u)+\rg(v)$.
\item Soient $F$ un sous-espace vectoriel de $E$, $G$ et $H$ deux
  supplémentaires de $F$. On note $p$ (\emph{resp.} $q$) la projection
  sur $F$ (sur $H$) parallèlement à $G$ (à $F$).

  Montrer que $\rg(p+q)=\rg(p)+\rg(q)$.
\end{enumerate}
   %exo 569 RMS 134-1
\end{exercice}

\begin{exercice}[concours=Mines-Ponts]
Déterminer les parties $G$ de $\M_n(\R)$ telles que $(G,\times)$ soit
un groupe multiplicatif et $G$ ne soit pas un sous-groupe de $\GL_n(\C)$.
   %exo 570 RMS 134-1
\end{exercice}


\begin{exercice}[concours=Mines-Ponts]
Soit $G$ un sous-groupe fini de $\GL_n(\C)$. Montrer que $\dsum_{M\in
  G}\tr(M)$ est un entier divisible par le cardinal de $G$.
   %exo 571 RMS 134-1
\end{exercice}

\begin{exercice}[concours=Mines-Ponts]
  \begin{enumerate}[\bfseries a)]
  \item Soit $G$ un sous-groupe fini de $\GL_n(\R)$ tel que
    $\dsum_{g\in G}\tr(g)=0$. Montrer que $\dsum_{g\in G}g=0$.
  \item Soient $G$ un sous-groupe fini de $\GL_n(\R)$ et $V$ un
    sous-espace vectoriel de $\R^n$ stable par tous les éléments de
    $G$. Montrer que $V$ admet un supplémentaire stable par tous les
    éléments de $G$.
  \end{enumerate}
   %exo 572 RMS 134-1
\end{exercice}


\begin{exercice}[concours=Mines-Ponts]
Déterminer les idéaux bilatères de $\M_n(\R)$, c'est-à-dire les
sous-groupes additifs stables par multiplication à gauche et à droite
par n'importe quel élément de $\M_n(\R)$.
   %exo 573 RMS 134-1
\end{exercice}

\begin{exercice}[concours=Mines-Ponts]
Soient $E$ un $\R$-espace vectoriel, $f$ et $g$ deux éléments de
$\Lin(E)$ tels que $fg-gf=id_E$.
\begin{enumerate}[\bfseries a)]
\item Montrer que $E$ est de dimension infinie ou nulle.
\item Montrer que $f$ n'est pas nilpotent.
\item Donner un exemple de triplet $(E,f,g)$ vérifiant les conditions
  précédentes.
\end{enumerate}
   %exo 574 RMS 134-1
\end{exercise}
\begin{solution}
  \begin{enumerate}[\bfseries a)]
  \item Si $E$ est de dimension finie $n\neq 0$, on a
    $n=\tr(id_E)=\tr(fg-gf)=0$, absurde.
  \item On a $f^ng-gf^n=nf^{n-1}$ par récurrence, ce qui donne une
    contradiction si $f$ nilpotente d'indice $n$.
  \item Mis du temps à trouver ! On considère $E:\K[X]$, $f:P\mapsto
    P'$ et $g:P\mapsto XP$, de sorte que $gf(X^n)=nX^n$ et
    $fg(X)=(n+1)X^n$.
    
    Remarque: dans le cas où $E$ est normé, $f$ et $g$ continus
    n'existent pas: en notant $\|\cdot\|$ la norme d'opérateur, on aurait
    $n\|f^{n-1}\|=\|f^ng-gf^n\|\le \big(\|fg\|+\|gf\|\big)\|f^{n-1}\|$
    et donc $\|fg\|+\|gf\|$ non borné, absurde.
  \end{enumerate}
\end{solution}
\end{enonce}