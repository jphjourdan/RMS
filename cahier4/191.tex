\begin{enonce}
\begin{exercise}[ID={Bloc Vuibert4 191},subtitle={},tags={}]
Dans le plan euclidien rapporté à un repère orthonormé, une ellipse $\cE$ a pour équation
$\frac{x^2}{a^2}+\frac{y^2}{b^2}-1=0$ ; donner une condition nécessaire et suffisante sur $(u,v,w)$ pour que la droite d'équation $ux+vy+w=0$ soit tangente à $\cE$.

Donner une condition nécessaire et suffisante sur $M_0(x_0,y_0)$ pour que les deux tangentes à $\cE$ passant par $M_0$ soient orthogonales.
\end{exercise}
\end{enonce}
