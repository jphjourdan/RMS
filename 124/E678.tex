\begin{enonce}
\begin{exercise}[ID={RMS124 E678 Mines-Ponts PSI},subtitle={2013 Mines-Ponts PSI},tags={}]
Soit $f:\left[0,+\infty\right[\to\R$ telle que $\int_0^{+\infty} f(x)\d x$ soit convergente.
\begin{enumerate}
  \item Si $f(x)$ admet une limite $\ell$ quand $x$ tend vers $+\infty$, que vaut $\ell$~?
  \item Donner un exemple où $f(x)$ n'a pas de limite quand $x$ tend vers $+\infty$.
  \item Montrer que si $f$ est décroissante, alors $xf(x)$ tend vers $0$ quand $x$ tend vers $+\infty$.
\end{enumerate}
\end{exercise}
\begin{solution}
\begin{itemize}
\item[3.] Par décroissance de $f$ : $0 \leq x f(x) \leq 2 \int_{x/2}^x f(t)\d t$.
\end{itemize}
\end{solution}
\end{enonce}
